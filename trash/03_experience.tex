\section{Our Experience: Projects and Workshops}
\label{sec:experience}

This section describes our relevant workshop experience, consisting of 8 visualization projects that used 15 creativity workshops, summarized in Tab.~\ref{tab:projects} and Tab.~\ref{tab:workshops} respectively, as well as 2 participatory and creative workshops with a variety of domain specialists at the World's leading visualization conference --- IEEE Vis~\cite{Rogers2016,Rogers2017}. As we analyzed more data than appeared in the resulting publications, including workshop artifacts and experiential knowledge, we refer to projects and workshops by unique identifiers throughout this paper, e.g.,~[\ref{pro:edina}] and [\ref{wor:edina}]. 

\subsection{Projects}

Our analysis makes sense of diverse projects that used workshops --- in addition to spanning 8 distinct domains, the project goals ranged from documenting and exploring the role visualization within a domain [\ref{pro:edina},~\ref{pro:eon},~\ref{pro:htva}], to creating tools that support existing analysis needs [\ref{pro:graffinity},~\ref{pro:cp}~\ref{pro:lineage}], to exploring the possibilities for funded collaboration [\ref{pro:updb},\ref{pro:arbor}]. Given different goals, the projects exhibit diverse results as a majority resulted in contributions to our collaborators domain as well as the visualization community [\ref{pro:edina},~\ref{pro:eon},~\ref{pro:htva},\ref{pro:graffinity},~\ref{pro:cp},~\ref{pro:lineage}], one project resulted in a funding proposal [\ref{pro:arbor}], and one project we consider to be a failure [\ref{pro:updb}]. Furthermore, the projects were completed on three continents, conducted by  researchers at City University London [\ref{pro:edina},~\ref{pro:eon},~\ref{pro:htva}], the University of Utah [\ref{pro:graffinity},~\ref{pro:lineage},~\ref{pro:updb},~\ref{pro:arbor}], and Monsash University [\ref{pro:cp}]. The diversity of our projects, in terms of their location, domain collaborators, and outcomes demonstrates that creativity workshops are a valuable method for visualization researchers and contributes to the potential transferability of our analysis.

Although we analyzed our collective experience, each co-author was involved in specific projects and we characterize our involvement in each project as either a primary or supporting researcher. We define the {\bf primary researcher} as the individual responsible for deciding to use a workshop, executing the workshop, and integrating the workshop results into a collaboration through analysis and action. Alternatively, the {\bf supporting researchers} are involved in the project to provide guidance or support. Our involvement includes serving as the primary researcher [\ref{pro:edina},~\ref{pro:eon},~\ref{pro:htva},~\ref{pro:graffinity},~\ref{pro:cp}] as well as secondary researchers [\ref{pro:lineage},~\ref{pro:updb},~ \ref{pro:updb}]. Thus, our experiential knowledge includes both leading and advising projects --- both of which are valuable perspectives for understanding the role of workshops in applied research. 

\subsection{Workshops}

We characterize our workshop experience by examining the role of workshops in the visualization design process and describing the character roles of stakeholders involved in the workshops.

Although design is inherently messy and the actions of designers are hard to describe, we abstract and simplify the intended outcome of each workshop by retrospectively  categorizing it by which \emph{design activities} from the design activity framework~\cite{McKenna2014} it fulfills as shown in Tab.~\ref{tab:workshops}. This extends and reinforces the terminology of Goodwin et al.~\cite{Goodwin2013}, which we call the workshop focus. {\bf Requirements workshops} generate an early understanding of user needs, often before significant efforts to create or develop prototypes [\ref{wor:edina}--\ref{wor:arbor}]. {\bf Design workshops} either generate design ideas to guide development~[\ref{wor:eon:des1}], or engage collaborators to evaluate designs ideas and prototypes~[\ref{wor:edina:des}--\ref{wor:htva:des}]. {\bf Evaluation workshops} present and evaluate final prototypes, often to conclude a project [\ref{wor:edina:eva}--\ref{wor:htva:eva}].

The boundaries between workshop focuses are nebulous, and, to some extent, all workshops could be considered requirements workshops because visualization research is about exploring how visualization could be used in specific domains. But we differentiate between requirements, design, and evaluation workshops to analyze workshops with similar objectives and context as the methods used during a workshop are influenced by the existence of prototypes, the amount of development work that has gone into a project, and the amount of time remaining for a collaboration. Requirements workshops, for example, focus on both exploring possible uses for visualization within a domain and eliciting tacit knowledge about a domain from collaborators. Design workshops either generate design ideas or elicit feedback on existing prototypes. Evaluation workshops typically conclude a project and evaluate the potential impact of visualization ideas.

In addition to characterizing workshops by design activity, they can be described by the stakeholders who are involved, such as the number of {\bf facilitators}, who guide and document the workshop execution, as well as the number {\bf participants}, who actually carry out the workshop methods. We categorize stakeholders as being visualization researchers, domain collaborators, or professional workshop facilitators. Generally, facilitators are visualizations researchers assisted by professional facilitators or domain collaborators. In requirements workshops and evaluation workshops, participants tend to be domain collaborators whereas design workshops may involve visualization researchers. Evaluation workshops again include domain collaborators. 

Initial analysis of our experience reveals some general characteristics of the scale and duration of requirements workshops. Our workshops range from a few hours to a few days, although a majority of our workshops were about one working day in length. In general, our workshop experience involves 1 - 4 facilitators guiding 5 - 17 participants through structured creativity methods. 

{\it The framework presented in this paper was developed for creativity requirements workshops which are used in the early formative stages of applied research projects.} While the goal of applied research projects may range from long term exploration and ideation to near term development and deployment of tools, analyzing our experience reveals common traits of creativity requirements workshops that are independent of the overall project goal. And, we identify areas where workshops can be tailored to the specific goals, time frames, and intended results. % Furthermore, our discussion, in Sec.~\ref{sec:discussion}, speculates on the generalizability of our contributions to workshops focuses beyond requirements. Next, we describe the research process used to generate actionable insights from experiential knowledge.


% \ek{This section needs to be rewritten. It's currently organized by time, but there are important constructs buried in the timeline e.g., the workshop focus and the workshop participants. I think we need to split it into two subsections: 1) workshop experience (with a pithy time-based description of our workshops); and 2) research process (with a description of the workshop constructs that we have developed and improved --- the focus, the character roles, and some general description of workshop size and duration). We need to discuss and agree on some of the terminology for \# 2.}

% This paper results from a multiyear, cross-institution, international collaboration in which we have analyzed how and why to use creativity workshops in applied visualization research. This paper's contributions arise from {\it reflection}, the analysis of experiences to generate insights~\cite{Boud1985,Sedlmair2012}. Specifically, we applied {\it critically reflective practice}~\cite{Brookfield1998}, a methodology of \emph{``synthesizing experience, reflection, self-awareness and critical thinking to modify or change approaches to practice}''~\cite{Thompson2008}. This methodology enabled us to collaboratively analyze our experience --- 8 visualization projects that used 15 creativity workshops, summarized in Tab.~\ref{tab:projects} and Tab.~\ref{tab:workshops} respectively, as well as 2 participatory and creative workshops with a variety of domain specialists at the World's leading visualization conference --- IEEE Vis~\cite{Rogers2016,Rogers2017}. As we analyzed workshop data beyond what appeared in resulting publications, including workshop artifacts and experiential knowledge, we refer to projects and workshops by unique identifiers throughout this paper, e.g.,~[\ref{pro:edina}] and [\ref{wor:edina}]. Although the remainder of this section summarizes our collaborative research process that intertwined analysis with action, our supplemental material contains an audit trail of documents produced throughout our analysis that illustrate significant reflective events. \ek{Audit trail draft is here: http://bit.ly/2Brn6er}

% Before our collaboration, we gained diverse workshop experience in the aforementioned design studies with colleagues at different organizations working with data sets in different domains: GIS [\ref{pro:edina}], energy~[\ref{pro:eon}], and defense [\ref{pro:htva}]. These projects were conducted by researchers at the same institution, \jd{DISCUSS: Not clear to me what this next clause means. Perhaps cut `distinct collaborators' given minor re-write above as `collaborator' is being used here for City/UT. So what is `varying levels of involvement'?} but with varying levels of involvement and distinct collaborators. Based on the success of these workshops, we separately used a creative requirements workshop at a different institution in a design study with neuroscientists [\ref{pro:graffinity}].

% Our collaboration started through informal discussions to answer a seemingly simple question about two workshops that used similar methods~[\ref{wor:eon},~\ref{wor:graffinity}]: what could we do better next time? Results from this discussion influenced how we ran an additional workshop using the same methods [\ref{wor:cp}]. %\sg{This makes the order of P4 and P5 seem odd - can we switch them as P5 was run before P4?} \ek{fixed} 
% To make sense of the three similar workshops [\ref{wor:eon},~\ref{wor:graffinity},~\ref{wor:cp}], we focused on their details, including the venue, participants, method length, and resulting artifacts. But workshop details could not account for profound differences in the projects, such as the reason for running workshops. Focusing on the details also limited the scope of our analysis as \jd{This next sentence needs some unpicking -- do you mean ``the reports of workshops used in similar stages of the design process were not sufficiently detailed to enable us to understand the rational for their differences, or the extent to which these differences had achieved what was intended. As such those who were not involved in the workshop itself were unable to learn about this potentially useful method from the reported experiences of those using of the workshops.''} we could not understand workshops that used different methods [\ref{wor:edina}, \ref{wor:htva}]. 

% Over many months, the scope our analysis expanded as we tried to understand how and why to use visualization creativity workshops. We searched existing literature on creativity and workshops~\cite{Biskjaer2017,CreativeEducationFoundation2015,DeBono1983,Gordon1961,Hamilton2016,Miller1989,Nickerson1999,Osborn1953,Sawyer2006,Sawyer2003,Shneiderman2005}, but this work did not directly apply to visualization as, for example, it did not explain the different uses of visualization creativity workshops. To address this, we categorized workshops by their corresponding \emph{design activities} from the design activity framework~\cite{McKenna2014}, as in Tab.~\ref{tab:workshops}. We used the design activity framework because: 1) it categorizes methods by their intended outcomes; 2) it recognizes that one method may fulfill multiple activities; 3) it does not prescribe a rigid process of activities; and 4) it maps to existing decision models~\cite{Munzner2009} and process models~\cite{Sedlmair2012}. Our categorization of workshops \new{helps to demonstrate the slight differences in focus when projects involve a series of workshops throughout the design process i.e., [\ref{pro:edina} -- \ref{pro:htva}].} Using the terminology of Goodwin et al.~\cite{Goodwin2013}, the three workshop focuses are: requirements workshops (fulfill the \emph{understand} and \emph{ideate} activities), design workshops (\emph{ideate} and \emph{make}), and evaluation workshops (\emph{make}, but focused on evaluating designs as part of the ongoing design process \jd{@MM cite PIsenberg on eval throughout design again here}). While our analysis and framework focus on requirements workshops, all of our workshop experience contributes to it.

% \jd{This is the bit where I think we have an inconsistency and we either need to ignore it and plough (US: plow?) on, or do something. I think that all CITY workshops were `requirements' workshops. We were trying to develop prototypes to get participants to think about what they might want to do in subsequent designs. We were not trying to produce working software in any of these cases. The three phases were intended to give us time to and to \emph{ideate} / \emph{make} and give others time to incubate ideas.
% I think the key thing here is that if you want to have interactive prototypes that involve data, you need to \emph{make} ad you can't do this in a day, so at present you need a series of workshops with a break*
% And I would \emph{recommend this} -- use a series of workshops to give participants time to incubate ideas and develop interactive designs that prompt further thinking -- it's what we do at CITY. At the end of 3 workshops we only ever get to a series of design ideas or concepts to inform further requirement building. 
% So the CITY three-workshop chronology is less fixed to a software development / process model than I think UT imagines! \emph{Discuss | Ignore}? 
% \\
% *High level languages like vega-lite may change this! Discussion ...}

% \ek{@JD - I'm trying to clarify the paragraph above. To discuss...
% \\ We categorized workshops by their corresponding \emph{design activities} from the design activity framework~\cite{McKenna2014}, as in Tab.~\ref{tab:workshops}. We used this framework because it allows us to retrospectively categorize workshops based on their motivation and outcomes. Workshops that \emph{``gather, observer, and research available information to find the needs of the user''} are categorized as fulfilling the \emph{understand} activity, workshops to create \emph{``ideas for supporting [user needs]''} or evaluate those ideas fulfill the \emph{ideate} activity, workshops to \emph{``concretize ideas intro tangible prototypes''} or gather feedback on prototypes fulfill the \emph{make} activity, and workshops that bring a prototype to support real-world data analysis fulfill the \emph{deploy} activity. Because design methods often have outcomes that unpredictable or difficult to precisely characterize, we differentiate between explicitly-focused activities and more emergent activities fulfilled by workshops. Our categorization helps to demonstrate the slight differences in workshop focuses introduced by Goodwin et al.~\cite{Goodwin2013}: requirements workshops (primarily fulfill the \emph{understand} activity, but may also support \emph{ideate}); design workshops (\emph{ideate} and \emph{make}); and evaluation workshops (\emph{make}, but focused on evaluating designs as part of an ongoing process) We recognize, however, that there are often significant design actions that are not captured by the workshops in a project, as, e.g., the time between design workshops [\ref{wor:eon:des1},~\ref{wor:eon:des2}] often involves time to develop interactive systems.}

% \sg{if (d)eploy is not seen in any workshop do we actually need it in the table? (also not sure why some single and some double vertical lines)} \ek{Fixed the lines. I've included deploy in the table for completeness of the DAF. And to show that there are unexplored possibilities for workshops. I added a note about this in the discussion. Maybe we should hint at it here too.} \sg{okay. So having thought a bit about these CITY projects being a bit different. Perhaps we just need something about the final goal of the vis research? The city ones were not about deploying the final prototypes but about showing really what opportunities there were in vis for their data. I think the others were more about getting the vis into current software/creating new software? (deployment) I think this can be shown in Table 2 quite easily without too much hassle}

% The existing workshop frameworks also failed to account for the typical characters of design studies --- visualization researchers and domain collaborators who must work together to explore visualizations for a domain problem~\cite{Sedlmair2012}. One existing framework, Creative Problem Solving~\cite{CreativeEducationFoundation2015}, identifies a client who is responsible for the problem, a facilitator who is responsible for managing the workshop, and a resource group who participate in the workshop. But, in visualization, one person is not responsible for the problem --- instead, design studies are about finding problems that evolves through the course of a collaboration. Thus, we developed our own description of who is involved in workshops. The person who makes the decision to run a workshop is the \emph{primary researcher}, they are typically responsible for the life cycle of the workshop and likely the first author on resulting visualization publications.
% \jd{Perhaps -- but isn't the key thing that they are the person who looks for opportunities for research in all of this and manages the research process -- with writing at the end? In Sec. 4 you describe this as ``adapting (the workshop) to explore emergent (research?) problems'' -- which seems key to this.} 
% The primary researcher assembles a workshop team consisting of \emph{facilitators} who will design and guide the workshop. The workshop \emph{participants} are typically domain collaborators [\ref{wor:edina},~\ref{wor:eon},~\ref{wor:graffinity}], but may also include visualization researchers [\ref{wor:htva},~\ref{wor:lineage},~\ref{wor:arbor}].

% As we compared and contrasted our experience and the workshop literature, we discovered that although workshops can vary on many attributes --- methods, participants, purpose, outcomes, etc. --- they tend to follow a similar process as there are common actions and decisions that occur before, during and after a workshop. We created an initial process model, describing the steps of using a workshop, based on literature and our experience. We continued to develop and generalize the model by reflecting on previous workshops [\ref{pro:edina},~\ref{pro:htva}] as well as running new workshops with different methods, participants, and outcomes. Working with colleagues outside of the collaboration, we assisted in the design and facilitation of half day workshops to understand the needs of genealogists~[\ref{wor:updb}] and psychiatrists~[\ref{wor:lineage}]. We created a two day workshop to jump-start a collaboration with biologists~[\ref{wor:arbor}]. And we were involved in the design and facilitation of the collaborative and creative Discovery Jam workshop~\cite{Rogers2016,Rogers2017}. We attempted to codify the outcomes of our thinking many times throughout this period, sometimes individually and sometimes collaboratively, in both narrative and diagram form. Writing often identified shortcomings in our thinking as well as useful refinements to our understanding. \jd{Feels like we need to make reference to something here -- the Google doc?}

% \jd{This next para seems very important, but I don't really understand it. @EK to explain}
% One such shortcoming was that analyzing the workshop process confounded the actions performed by researchers in the context of the collaboration with the actions performed by participants in the context of the workshop. We split out analysis into separate, but still interconnected, concepts: the workshop process and workshop design. This allows us to examine similarities in the process of workshops that use different methods (e.g., [\ref{wor:edina}] and [\ref{wor:eon}]). It also supports analyzing the reuse of workshop methods across different projects (e.g., [\ref{wor:eon}] and [\ref{wor:graffinity}]). And, it enables us to leverage extensive research on creativity methods~\cite{Biskjaer2017,Grube2008,McKenna2014,Sanders2005,Sanders2010} within workshop design. The resulting framework from this analysis is described next.