\section{Motivation and Background}
\label{sec:background}

In our experience, \workshops provide tremendous value to applied visualization stakeholders --- researchers and the domain specialists with whom they collaborate. \workshops provide time for focused thinking about a collaboration, which allows stakeholders to share expertise and explore visualization opportunities. In feedback, one participant reported the workshop was {\it  ``a good way to stop thinking about technical issues and try to see the big picture''}~\cite{Goodwin2016}.

\workshops can also help researchers understand analysis pipelines, work productively within organizational constraints, and efficiently use limited meeting time. As another participant said: {\it ``The structured format helped us to keep on topic and to use the short time wisely. It also helped us rapidly focus on what were the most critical needs going forward. At first I was a little hesitant, but it was spot-on and wise to implement''}~\cite{Lisle2017}.

Furthermore, \workshops can build trust, rapport, and a feeling of co-ownership among project stakeholders. Researchers and collaborators can leave workshops feeling inspired and excited to continue a project, as reported by one participant: {\it ``I enjoyed seeing all of the information visualization ideas \ldots very stimulating for how these might be useful in my work''}~\cite{Goodwin2016}.

Based on these reasons, our view is that \workshops have saved us significant amounts of time pursuing problem characterizations and task analysis when compared to traditional visualization design approaches that involve one-on-one interviews and observations. What may have taken several months, we accomplished with several days of workshop preparation, execution, and analysis. In this paper we draw upon 10 years of experience using and refining workshops to propose a framework that enables others to use \workshops in the future.

\workshops are based on workshops used for software requirements and creative problem-solving~\cite{Goodwin2013}. Software requirements workshops elicit specifications for large-scale systems~\cite{Jones2007} that can be used in requirements engineering~\cite{Jones2005} and agile development~\cite{Hollis2013}. There are many documented uses of such workshops~\cite{Jones2008,Maiden2004,Maiden2007,Maiden2005}, but they do not appropriately emphasize the mindset of visualization researchers or a focus on data and analysis.

More generally, creative problem-solving workshops are used to identify and solve problems in a number of domains~\cite{Osborn1953} --- many frameworks exist for such workshops~\cite{CreativeEducationFoundation2015,DeBono1983,Gordon1961,Gray2010,Kumar2012}. Meta-analysis of these frameworks reveal common workshop characteristics that include: promoting trust and risk taking, exploring a broad space of ideas, providing time for focused work, emphasizing both problem finding and solving, and eliciting group creativity from the cross-pollination of ideas~\cite{Nickerson1999}. 

%The characteristics of creative problem-solving workshops overlap with the practices of applied visualization that include: establishing rapport with collaborators~\cite{Shneiderman2006}, exploring a broad space of possible designs~\cite{Sedlmair2012}, and recognizing that designs are closely linked to problem formulation~\cite{Munzner2009}. This paper is, in part, about adopting and adapting creative problem-solving workshops for visualization. 

Existing workshop guidance, however, does not completely describe \workshops. The key distinguishing feature of \workshops is the explicit focus on visualization, which implies three {\bf visualization specifics} for effective workshops and workshop guidance:
\begin{itemize}[nolistsep,noitemsep]
    \item Workshops should promote a {\bf visualization mindset} --- the set of beliefs and attitudes held by project stakeholders, including an evolving understanding about domain challenges and visualization~\cite{McCurdy2016a,Sedlmair2012} --- that fosters and benefits an exploratory and visual approach to dealing with data while promoting trust and rapport among these stakeholders~\cite{Shneiderman2006};
    \item Workshops should contribute to {\bf visualization methodologies} --- the research practices of visualization, including process and decision models~\cite{McKenna2014,Munzner2009} --- by creating artifacts and knowledge useful in the visualization design process; and
    \item Workshops should use {\bf visualization methods} that explicitly focus on data visualization and analysis by exploring visualization opportunities with the appropriate {\it information location} and {\it task clarity}~\cite{Sedlmair2012}.
\end{itemize}
This paper is, in part, about adopting and adapting creative problem-solving workshops to account for these visualization specifics.


% Creativity is a complex phenomenon studied from many perspectives, including psychology~\cite{Sawyer2006}, sociology~\cite{Lubart1999}, and biology~\cite{Martindale1999}. Whether workshops can enhance creativity is an open question~\cite{Nickerson1999,Sawyer2006}: while some experiments show that group-based methods can reduce creativity~\cite{Bouchard1969,Mullen1991}, critics argue that these experiments lack ecological validity~\cite{Nolan2003}. Experimentally testing the relationship between workshops and creativity is beyond the scope of this paper. Instead, this paper focuses on understanding and communicating how we use \workshops in applied collaborations.