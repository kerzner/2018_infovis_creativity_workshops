\section{Process Model: Reflective Recommendations}
\label{sec:process}

While the prior two sections describe and illustrate the design of workshop methods, this section expands the scope of our reflective analysis to encompass the process of using workshops --- from deciding to run a workshop through acting on its results. Specifically, this section presents \reviseme{17} actionable recommendations, based in our experience and grounded in creativity literature, that identify a likely beneficial course of action for future workshops. They are summarized in Tab.~\ref{tab:recommendations} and described in the context of the workshop process model next.

\begin{table}
    \small
    \centering
    \begin{tabular}{|m{0.06\linewidth}|p{0.8\linewidth}|}
    \hline
    ID & Recommendation \\
    \hline
    \rec{rec:theme}{} & Articulate a mutually beneficial workshop theme. \\
    \rec{rec:participants}{} & Recruit diverse and creative participants. \\
    \rec{rec:diverse}{} & Recruit engaged and excited participants and facilitators. \\
    \rec{rec:vocab}{} & Evaluate domain knowledge before designing. \\
    \rec{rec:constraints}{} & Recognize workshop constraints to winnow the design space. \\
    \rec{rec:improvise}{} & Recognize tradeoffs of scripted vs improvised. \\
    \rec{rec:pilot}{} & Pilot workshops to test, evaluate, and improve methods. \\
    \hline
    \rec{rec:prepare}{} & Prepare for execution with facilitation resources. \\
    \rec{rec:materials}{} & Gather appropriate materials for workshop methods. \\
    \rec{rec:guidance}{} & Guide the workshop to foster creativity. \\
    \rec{rec:adapt}{} & Adapt facilitation to participant reactions. \\
    \rec{rec:artifacts}{} & Create artifacts --- anything not written will likely be lost. \\
    \hline
    \rec{rec:inspiring}{} & Expect rich and inspiring artifacts that warrant action. \\ 
    \rec{rec:time}{} & Allocate significant time for workshop analysis --- a day or more. \\
    \rec{rec:complement}{} & Complement workshops with traditional user-centered design methods.  \\
    \rec{rec:generative}{} & Use workshop outputs for divergent visualization design methods.  \\ 
    \rec{rec:evaluative}{} & Use workshop outputs for convergent visualization design methods.  \\
    \rec{rec:revisit}{} & Revisit workshop output and insights throughout the project. \\
    \hline
    \end{tabular}
    \caption{Summary of the workshop process recommendations.}
    \label{tab:recommendations}
\end{table}

\subsection{Decide \& Design}

The decision to use a workshop can be motivated by many factors, including: to deliberately and explicitly stimulate creativity in a project [\ref{pro:eon}]; to sample problems faced by analysts in different organizations [\ref{pro:cp}]; to explore shared needs from seemingly diverse analysts [~\ref{pro:graffinity},~\ref{pro:cp},~\ref{pro:lineage}]; to make use of limited meeting time with groups of collaborators [\ref{pro:edina},~\ref{pro:htva},~\ref{pro:arbor}]; and to identify surrogate data if real data are not available [\ref{pro:htva}]. More generally, workshops can help to establish rapport with analysts and to rapidly characterize domain challenges while exploring specific analysis needs. But the abstract reasons for running a workshop may not be actionable and often provide little incentive for collaborators to want to participate. 

Articulating a \emph{workshop theme}, a concise description of what will happen in the workshop, can identify an outcome that is beneficial to both collaborators and researchers while providing useful criteria to winnow the space of possible effective workshops [\ref{rec:theme}]. Themes can be narrowly focused, such as to explore specific domain challenges or visualization opportunities---\emph{``enhancing legends with visualizations''} [\ref{wor:edina}]. Themes can also be broadly focused, such as to explore challenges faced by analysts across a domain---\emph{``identify analysis and visualization opportunities for improved profiling of constraint programmers''} [\ref{wor:cp}]. The theme helps to winnow the space of possible workshops, even if it is seemingly broad. For example, our workshop, broadly themed with \emph{``explore shared user needs for visual analysis of retinal connectomes,''} was still focused on a specific area of our collaborator's workflow --- connectome analysis. We could have used a broader theme, for example, to examine the challenges associated with generating, storing, and retrieving their data. The exact process of creating a theme varies by researcher, but it usually identifies a reason for collaborators and researchers to attend the workshop while accounting for constraints such as the project goal, duration, and funding.

As the theme winnows the focus of the workshop, it serves as a step toward understanding the required knowledge about a domain as we intend to ask interesting questions related to the theme, about domain challenges, relevant data, and specific analysis tasks and tools [\ref{rec:vocab}]. In some cases, we had sufficient command of the domain to start designing workshops quickly [\ref{wor:edina},~\ref{wor:htva},~\ref{wor:eon}]. In other cases, we have used traditional user-centered design methods, interviews and contextual inquiry, to iteratively improve our theme and to learn about analysis challenges relevant to it [\ref{wor:graffinity},~\ref{wor:lineage}]. Alternatively, when time is limited or domain vocabulary complex, we have recruited domain specialists to assist in developing the workshop theme and designing a workshop around that [\ref{wor:cp},~\ref{wor:arbor}]. The understanding of the domain problem and the workshop theme can evolve together, as we learn more about the domain by designing the workshop.

The theme influences the ideal workshop participants. Workshops with narrow themes may benefit from frontline analysts as participants, as in our workshop themed with \emph{``understand the main tasks of psychiatric researchers''} [\ref{wor:lineage}], where we invited research analysts. If this workshop focused on a broader theme, we may have invited psychiatric clinicians or support staff to participate. When working with broader themes, we have recruited participants who can contribute diverse perspectives about the domain challenges. For example, we have used an online survey of practitioners, students, and educators to identify potential participants. Those with the most interesting or creative responses who expressed their interest in continued participation were shortlisted [\ref{wor:cp}]. The survey had an added benefit of collecting relevant domain knowledge, that was used to tailor the workshop methods. When working within an organization, we have recruited diverse members of the research lab --- including graduate students, support staff, and professors [\ref{wor:graffinity}]. Yet, there have also been cases where we had little control over the participants, such as collaborators who identified participants as a workshop constraint [\ref{wor:eon},\ref{wor:htva}]. Across all workshops, recruiting diverse and creative participants may contribute to the successful workshop execution and outcomes as they contribute experiential knowledge that enables exploration of a broader ideaspace [\ref{rec:diverse}]. 

On a practical note, the participants as well as workshop facilitators should be committed to focus on the workshop without distractions, such as leaving for a meeting or checking e-mail [\ref{rec:participants}]. In our experience, individuals in the workshop who are distracted by communication from beyond the workshop, or who are passively observing workshop methods, have distracted from effective execution [\ref{wor:graffinity},~\ref{wor:arbor}]. Clearly communicating the workshop expectations in workshop invitations can aid in recruiting participants and facilitators. While expectations of focused thinking should be reinforced during the workshop opening (e.g., \emph{switch off all electronic devices}), they should be communicated upfront to filter potential participants.

Deciding to use a workshop also involves identifying relevant constraints, including the duration and location, in order to further winnow the possible design space [\ref{rec:constraints}]. We have found one day (6 - 8 hours) sufficient for workshops [\ref{wor:edina},~\ref{wor:eon},~\ref{wor:htva},~\ref{wor:graffinity},~\ref{wor:cp}], half day workshops can work, but may feel rushed and do not allow for sufficient incubation and iteration [\ref{wor:lineage},\ref{wor:updb}], and two days [\ref{wor:arbor}], although productive, would be a large commitment from collaborators. 
%\jd{Here is where we could say something about the linked workshops used in the three early CITY cases -- I do think we need to mention this explicitly -- we used this structure to give us time to develop digital interactive examples that allowed us to explore design possibilities and requirements.} \ek{We will talk about this. Not sure where yet.} 
Another constraint is the location and venue. Creativity literature expounds the importance of neutral, well-lit venues~\cite{CreativeEducationFoundation2015,Isaksen2000}. While such venues can be successful [\ref{wor:eon},~\ref{wor:htva}], we have also had success hosting workshops in on-site conference rooms  [\ref{wor:graffinity},~\ref{wor:cp},~\ref{wor:lineage}]. The venue affordances, such as the room size and physical layout, are important factors in designing the workshop. 

% It can be useful to identify the theme, participants, facilitators, and constraints before designing the workshop as the design must fulfill these criteria. The exact process of designing a workshop is difficult to describe. It varies by researcher and is learned through experience. For example, we spent approximately 25 hours designing and piloting the one of our first requirements workshop [\ref{wor:eon}], but this reduced to about 5 hours when reusing the same basic workshop structure but fine-tuned for a different domain [\ref{wor:cp}]. The workshop design considerations, described in Sec.~\ref{sec:design}, and the illustrative example workshop structure described in Sec.~\ref{sec:example} are useful starting points for designing workshops. % \sg{I tried but probably grossly underestimated both. Difficult to quantify in hours (I did this in people hours i.e. 2hrs for 1hr with 2 people). eon involved 3 (JD, SG, AD) - 5 (SJ, GD) people hours - various meetings, prepping lots of material + 2 pilots with a few more participants. CP was my prep time with 2 domain experts as co-facilitators in 1 or 2 meetings to check understanding, methods and terminology of prompts + brainstorming and exploring changes to structure incl. a call with SJ+GD.} 

Designing workshops depends on the preferences and experiences of the facilitator, ranging from more scripted to more improvisational [\ref{rec:improvise}]. These preferences appear in our workshop experience, as we have improvised and adapted to the changing environment, for example, when participants took initiative and proposed a method that they would find helpful [\ref{wor:htva}]. On the other hand, other workshops relied on more formal planning and testing of methods before the workshop [\ref{wor:eon},~\ref{wor:graffinity},~\ref{wor:arbor}]. \ek{This may connect to sec.~\ref{sec:design}. Need to fix it.}

Piloting workshop methods or the entire workshop can help to test, evaluate and improve workshop methods [\ref{rec:pilot}]. We have used pilots to test how understandable are methods [\ref{wor:eon},~\ref{wor:graffinity}]; to evaluate whether method prompts create interesting results [\ref{wor:lineage},~\ref{wor:arbor}]; and to find errors in method prompts and materials [\ref{wor:eon},~\ref{wor:graffinity},~\ref{wor:lineage},~\ref{wor:arbor}]. This often involves the use of proxy workshop participants, such as visualizations researchers~[\ref{wor:eon}] or domain collaborators~[\ref{wor:arbor}] --- providing an opportunity to improve our understanding of the domain challenges.  Importantly, piloting the workshop with all of the facilitators can influence its smooth execution. Even when reusing a known workshop structure tailored to a new domain or with a new facilitator team, a pilot would have allowed the team to be better prepared for their individual roles on the day as well as gain a better understanding of the methods and expected outcomes [\ref{wor:cp}].
%\ek{SG - can you add something here? Possible failure case with [\ref{wor:cp}]?}

\subsection{Execute \& Adapt}

Before execution, we prepare to facilitate the workshop and prepare practical considerations, such as the materials and venue. Reviewing the guidelines of effective workshop facilitation from the literature (e.g., \cite{Brooks-Harris1999,CreativeEducationFoundation2015,Hamilton2016,Macanufo2010,Stanfield2002}), reveals common principles to keep in mind while executing the workshop, including: being energized, providing encouragement, demonstrating acceptance, using humor, and being punctual [\ref{rec:prepare}]. 

Gathering the correct materials is also important [\ref{rec:materials}] --- we have mistakenly bought post-it notes that are too big, causing participants to write more than one idea on a sheet and making it challenging to use methods that involve sorting or ranking ideas. Preparing the venue is also important, as the furniture arrangement should promote a feeling of co-ownership and encourage participation --- a semi-circle seating arrangement works well for this~\cite{Vosko1991}. A mistake in one of our workshops was to have the speaker using a podium, which implied a hierarchy between facilitators and participants, hindering communication~\cite{Rogers2016}.

During execution the workshop team must guide participants through the methods, allowing for exploration but moving toward the common goal [\ref{rec:guidance}]. Conversations that deviate from the day's focus should be redirected, but this requires careful judgment to determine whether a conversation is likely to be fruitful and sensitivity about the creative atmosphere. When allowed to discuss freely, participants commented \emph{``we had a tendency to get distracted [during discussions]''}[\ref{wor:graffinity}]. Whereas more active guidance resulted in feedback: \emph{``we were guided and kept from going too far off track despite our tendencies to do so. This was very effective''} [\ref{wor:arbor}]. Yet, redirection can be jolting and can contradict some of the agreed guidelines (e.g., \emph{``all ideas are valid!''}). It may be beneficial to prepare participants for redirection with another guideline during the workshop opening: \emph{``facilitators may keep you on track gently, so please be sensitive to their guidance.''}

As facilitators guide the workshop, they can interpret group dynamics to adapt to the changing situation [\ref{rec:adapt}]. If participants do not find a method helpful, they may propose their own as when analysts proposed walking through visualization analysis scenarios in place of a planned method [\ref{wor:htva}]. Facilitators should be prepared for flexibility, perhaps by having alternative methods planned or by being ready to improvise. It requires judgment to deviate from the plan, and the design considerations should be considered on-the-fly as the workshop adapts to participant responses.
\ek{JD: what else can we say about this?}
\jd{@EK: to me it's all about the things we are trying to achieve and whether we have achieved them, so I am always on the lookout for this: diverse communication, levels of energy time, ownership, agency. I'd like this list of characteristics (does Sarah have it?) so that I can write a little about this here.}

Workshops produce a tremendous amount of information and discussions are ephemeral: anything not written down will likely be lost [\ref{rec:artifacts}]. In one case, audio recordings provided valuable information [\ref{wor:lineage}], but audio for longer workshops may not be useful as it requires tremendous time to transcribe and analyze after the workshop~\cite{Lloyd2011}. Recording may also hinder creativity as participants become self conscious\ek{SJ - do you have a citation for this?}. We make an effort to document all activities in the workshop, by note taking or through methods that create artifacts. The workshop team must know the expectations for note taking and pilot workshops will help with this. A pilot for [\ref{wor:cp}] for example, may have reduced the note taking pressure on the primary researcher during the day.

\subsection{Analyze \& Act}

Effective workshops generate rich and inspiring artifacts that can include hundreds of post-it notes, posters, sketches, and other items of documentation. While analyzing the artifacts leads to insights about the opportunities, needs, and concerns of participants, the analysis will warrant continued action as workshops are one part of an on-going design conversation between researchers and their collaborators [\ref{rec:inspiring}].

The primary researcher typically analyzes the workshop output soon after the workshop, usually within a day, so workshop discussions and (particularly any non-documented) ideas are fresh in memory. Allocating time for analysis is important because it can often require more time than the workshop itself [\ref{rec:time}]. We have started analysis by typing or photographing artifacts into documents or spreadsheets, allowing  us to become familiar with all ideas in the artifacts. This also enables sharing the output to enlist diverse stakeholders --- such as collaborators or other workshop team members --- in making sense of the results and clarifying ambiguous requirements. This is particularly important in domains with complex vocabulary.

The primary researcher is critically important in this stage as they decide how to analyze the workshop output and what to do with the results of that analysis. In our failed project [\ref{wor:updb}], we ran a workshop without clearly identifying the primary researcher, and workshop output went unused.

The specific analysis methods will depend on the form of the artifacts which is directly influenced by workshop methods. In most cases [\ref{wor:eon},~\ref{wor:graffinity},~\ref{wor:cp},~\ref{wor:lineage},~\ref{wor:updb}],  we used qualitative analysis methods -- open coding, mindmapping, and other less formal processes -- to group workshop artifacts into common themes or tasks. We often ranked these themes and tasks by various criteria, including, novelty, ease of development, potential impact on the domain, and relevance to the collaboration. Quantitative analysis methods should be approached with caution as the frequency of an idea provides little information about its novelty, usefulness, or potential impact. In other cases~[\ref{wor:edina},~\ref{wor:htva}], workshop methods generated specific requirements, tasks, or scenarios that could be editing for clarity and directly integrated into the design process. Regardless of the methods used, the results of requirements workshops can be framed in the vocabulary of the design activity framework~\cite{McKenna2014} as opportunities, constraints, and considerations. Furthermore, the ways in which we act on workshop insights can be characterized by their design activities.

The results of analysis can influence the understand design activity as they are used to scope traditional  user-centered design methods, such as interviews and contextual inquiry [\ref{rec:complement}]. For example, a common theme of output from our neuroscience workshop was to \emph{``analyze multi-hop relationships''} [\ref{wor:graffinity}]. Using this theme, we focused interviews on the challenges of analyzing connectivity, revealing low-level tasks that inspired subsequent prototypes.

The results of workshop analysis can influence the ideate and make design activities. For example, we have used the workshop output in parallel prototyping [\ref{wor:graffinity},\ref{wor:cp},], as well as to decide on features for in-development software tools~[\ref{wor:lineage}], as one of our collaborators who used the workshop told us \emph{``I personally got a much better understanding of what they were trying to do and what information they needed to do it ... which ultimately guided our design decisions.''} In other cases [\ref{wor:edina},~\ref{wor:eon},~\ref{wor:htva}], we have used the workshop output as input to additional workshops focused on rapidly exploring the possibilities for visualization design. These activities may adapt existing software to newly discovered analysis needs or explore entirely new visualization techniques as in our neuroscience project~\ref{wor:graffinity}, where the outputs inspired plugins for existing tools that we iteratively developed into a novel visualization technique. In all of these cases, our actions can be considered divergent --- expanding space of possible visualization designs currently being considered [\ref{rec:generative}].

Also within the ideate and make design activities, workshop results can be used in convergent design methods --- contracting the space of possible visualization designs [\ref{rec:evaluative}]. The workshop output can involve design considerations, such as reaching \emph{``everything in three clicks''} [\ref{wor:eon}] and providing \emph{``access [to] underlying database keys''} [\ref{wor:graffinity}] from visualizations. These criteria can be used to winnow the space of possibilities, for example, to evaluate, focus, and refine designs and prototypes. 

We emphasize that analyzing the output and acting on the results of analysis occur iteratively and that workshop output should be revisited throughout the project [\ref{rec:revisit}]. Workshop artifacts can provide valuable evidence about the contributions of applied work as they can document that visualization systems fulfill real analysis needs. They can also be used to document the evolution of ideas that occurs throughout design studies. \ek{JD/SG - do you have an example of how the workshop artifacts show the provenance of design decisions? Or how our understanding of the workshop artifacts can change throughout the collaboration?}

\sg{I find the order currently a little odd. In Sec 4: we introduce the process then design model. But we don't get to the process model details (sec 7) until after the design model (sec 5). Could we move Sec 7 before sec 5? Then we have overview of workshop process, detail of methods, then example workshop then discussion. If not, could we switch 4.1 and 4.2 around so it is a bit clearer that it'll come later?}