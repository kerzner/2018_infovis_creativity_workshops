\section{Research Process}
\label{sec:research}

The contributions in this paper arise from {\it reflection} --- the analysis of experiences to generate insights~\cite{Boud1985,Schon1988}. More specifically, we applied a methodology of {\it critically reflective practice}~\cite{Brookfield1998}, summarized by Thompson and Thompson~\cite{Thompson2008} as {\it ``synthesizing experience, reflection, self-awareness and critical thinking to modify or change approaches to practice.''}

We analyzed our collective experience and our \workshop data, which consisted of documentation, artifacts, participant feedback, and research outputs. The analysis methods that we used can be described through three metaphorical lenses of critically reflective practice:
\begin{itemize}[nolistsep,noitemsep]
\item The lens of our collective experience --- we explored and articulated our experiential knowledge through interviews, discussions, card sorting, affinity diagramming, observation listing, and observations-to-insights~\cite{Kumar2012}. We codified our experience, individually and collectively, in both written and diagram form. We iteratively and critically examined our ideas in light of workshop documentation and artifacts.
\item The lens of existing theory --- we grounded our analysis and resulting framework in the literature of creativity and workshops~\cite{CreativeEducationFoundation2015,Biskjaer2017,DeBono1983,Gordon1961,Hamilton2016,Miller1989,Nickerson1999,Osborn1953,Sawyer2003,Sawyer2006,Shneiderman2005} as well as visualization design theory~\cite{McKenna2014,Munzner2009,Sedlmair2010,Tory2004}. 
\item The lens of our learners (i.e., readers) --- in addition to intertwining our analysis with additional workshops, we shared drafts of the framework with visualization researchers, and we used their feedback to make the framework more actionable and consistent.
\end{itemize}

Our reflective analysis, conducted over two years, was messy and iterative. It included periods of focused analysis and writing, followed by reflection on what we had written, which spurred additional analysis and rewriting. Throughout this time, we generated diverse artifacts, including models for thinking about how to use workshops, written reflections on which methods were valuable to workshop success, and collaborative writing about the value of workshops. This paper's Supplemental Material contains a timeline of significant events in our reflective analysis and \numberOfAudits supporting documents that show how our ideas evolved into the following framework.