\section{Related Work}
\label{sec:related}

Workshops are commonly used in a number of fields, such as business~\cite{Gray2010,Hamilton2016,Stanfield2002} and education~\cite{Anderson2000,Brooks-Harris1999}. Guidance from these fields, however, does not emphasize the role of workshops in a design process, which is central to applied visualization. Therefore, we focus this section on workshops as visualization design methods.  

\workshops can be framed as a method for user-centered design~\cite{Norman1986}, participatory design~\cite{Muller1993}, or co-design~\cite{Sanders2008} because they involve users directly in the design process --- we draw on work from these fields that have characterized design methods. Sanders et al.~\cite{Sanders2010}, for example, characterize methods by their role in the design process. Biskjaer et al.~\cite{Biskjaer2017} analyze methods based on concrete, conceptual, and design space aspects. Vines et al.~\cite{Vines2013} propose ways of thinking about how users are involved in design. Dove~\cite{Dove2016} describes a framework for using data visualization in participatory workshops. A number of books also survey existing design methods~\cite{Buxton2010,Kumar2012} and  practices~\cite{Knapp2016,Laural2003,Sanders2013}. These resources are valuable for understanding design methods but do not account for visualization specifics such as methodologies that emphasize the critical role of data early in the design process~\cite{Lloyd2011}.

\workshops can also be framed within existing visualization design process and decision models~\cite{Marai2018,McKenna2014,Munzner2009,Sedlmair2012,Tory2004}. More specifically, \workshops focus on eliciting opportunities for visualization software from collaborators. They support the {\it understand} and {\it ideate} design activities~\cite{McKenna2014} or fulfill the {\it winnow}, {\it cast}, and {\it discover} stages of the design study methodology's nine-stage framework~\cite{Sedlmair2012}.

A number of additional methods can be used in the early stages of applied work. Sakai and Aert~\cite{Sakai2015}, for example, describe the use of card sorting for problem characterization. McKenna et al.~\cite{McKenna2015} summarize the use of qualitative coding, personas, and data sketches in collaboration with security analysts. Koh et al.~\cite{Koh2011} describe workshops that demonstrate a wide range of visualizations to domain collaborators, a method that we have adapted for use in \workshops as described in Sec.~\ref{sec:workshop-methods}. Roberts et al.~\cite{Roberts2016} describe a method for exploring and developing visualization ideas through structured sketching. This paper is about how to use these design methods, and others, within structured \workshops.

Visualization education workshops are also relevant to \workshops. Huron et al.~\cite{Huron2016} describe data physicalization workshops for constructive visualization with novices. He et al.~\cite{He2017} describe workshops for students to think about the relationships between domain problems and visualization designs. In contrast, we frame \workshops as a method for experienced researchers to pursue domain problem characterization. Nevertheless, we see opportunities for participatory methods, such as constructive visualization~\cite{Huron2014} and sketching~\cite{Walny2015}, to be integrated into \workshops.