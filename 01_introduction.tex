Two key challenges in the early stages of applied visualization research are to find pressing domain problems and to translate them into interesting visualization opportunities. Researchers often discover such problems through a lengthy process of interviews and observations with domain collaborators that can sometimes take months~\cite{Lam2012,McKenna2014,Sedlmair2012}. A number of recent projects, however, report on the use of workshops to characterize domain problems in just a few days of focused work~\cite{Dykes2010,Goodwin2013,Goodwin2016,Kerzner2017,Nobre2017,Walker2013}. More specifically, these workshops are {\bf creative visualization-opportunities workshops (\workshops)}, in which researchers and their collaborators explore opportunities for visualization in a domain~\cite{Goodwin2013}. When used effectively, such workshops reduce the time and effort needed for the early stages of applied visualization work, as noted by one participant: \emph{``The interpersonal leveling and intense revisiting of concepts made more progress in a day than we make in a year of lab meetings \ldots [the workshop] created consensus by exposing shared user needs''}~\cite{Kerzner2017}.

The \workshops reported in the literature were derived and adapted from software requirements workshops~\cite{Jones2007} and creative problem-solving workshops~\cite{CreativeEducationFoundation2015} to account for the specific needs of visualization design. These adaptations were necessary because existing workshop guidance does not appropriately emphasize three characteristics fundamental to applied visualization, which we term {\it visualization specifics}: the {\it visualization mindset} of researchers and collaborators characterized by a symbiotic collaboration~\cite{Sedlmair2012} and a deep and changing understanding of domain challenges and relevant visualizations~\cite{McCurdy2016a}; the connection to {\it visualization methodologies} that include process and design decision models~\cite{Munzner2009,Sedlmair2012}; and the use of {\it visualization methods} within workshops to focus on data analysis challenges and visualization opportunities~\cite{Goodwin2013}. 

The successful use of \workshops resulted from an ad hoc process in which researchers modified existing workshop guidance to meet the needs of their specific projects and reported the results in varying levels of detail. For example, Goodwin et al.~\cite{Goodwin2013} provide rich details, but with a focus on their experience using a series of workshops in a collaboration with energy analysts. In contrast, Kerzner et al.~\cite{Kerzner2017} summarize their workshop with neuroscientists in one sentence even though it profoundly influenced their research. Thus, there is currently no structured guidance about how to design, run, and analyze \workshops. Researchers who are interested in using such workshops must adapt and refine disparate workshop descriptions.

In this paper, we --- a group of visualization and creativity researchers who have been involved with every \workshop reported in the literature --- 
reflect on our collective experience and offer guidance about how and why to use \workshops in applied visualization. More specifically, this paper results from a 2-year international collaboration in which we applied a methodology of \emph{critically reflective practice}~\cite{Brookfield1998} to perform meta-analysis of our collective experience and research outputs from conducting 17 workshops in 10 visualization contexts~\cite{Dykes2010,Goodwin2016,Goodwin2013,Kerzner2017:utdb,Kerzner2017,Lisle2017,Nobre2017,Rogers2016,Rogers2017,Walker2013}, combined with a review of the workshop literature from the domains of design~\cite{Biskjaer2017,Dove2014,Kumar2012,Sanders2010}, software engineering~\cite{Horkoff2015,Jones2008,Jones2005,Jones2007,Maiden2010,Maiden2004,Maiden2005}, and creative problem-solving~\cite{DeBono1983,Gordon1961,Hamilton2016,Miller1989,Osborn1953}. 

This paper's primary contribution is a framework for \workshops. The framework consists of: 1) a process model that identifies actions before, during, and after workshops; 2) a structure that describes what happens in the beginning, in the middle, and at the end of effective workshops; 3) a set of \numberOfGuidelines actionable guidelines for future workshops; and 4) an example workshop and example methods for future workshops. To further enhance the actionability of the framework, in Supplemental Materials\footnote{\href{http://bit.ly/CVOWorkshops}{http://bit.ly/CVOWorkshops/}} we provide documents with expanded details of the example workshop, additional example methods, and \numberOfPitfalls pitfalls we have encountered when planning, running, and analyzing \workshops. 

We tentatively offer a secondary contribution: this work exemplifies critically reflective practice that enables us to draw upon multiple diverse studies to generate new knowledge about visualization in practice. Towards this secondary contribution we include, in Supplemental Materials, an {\it audit trail}~\cite{Carcary2009,Lincoln1985} of artifacts that shows how our thinking evolved over the 2-year collaboration. 

In this paper, we first summarize the motivation for creating this framework and describe related work in Sec.~\ref{sec:background} and \ref{sec:related}. Next, we describe our workshop experience and reflective analysis methods in Sec.~\ref{sec:experience} and ~\ref{sec:research}. Then, we introduce the framework in Sec.~\ref{sec:framework}~--~\ref{sec:after}. After that, we discuss implications and limitations of the work in Sec.~\ref{sec:discussion}. We conclude with future work in Sec.~\ref{sec:conclusion}.