\begin{tcolorbox}[floatplacement=t,float,title=Fostering Creativity with CACTI Factors]
Reflecting on our experience, and reviewing relevant literature~\cite{Nickerson1999,Osborn1953,Sawyer2003,Sawyer2006,Shneiderman2005}, reveals a number of key factors that influence the engagement and creativity of workshop participants: fostering, maintaining, and potentially varying the levels of collegiality, agency, challenge, trust and interest associated with each, as well as the focus on visualization and data in the context of the specialist domain. To help us remember these factors, we term them {\bf CACTI factors}: 
\begin{itemize}[noitemsep,nolistsep]
\item(\textbf{C})ollegiality -- the degree to which communication and collaboration are encouraged and occur;
\item(\textbf{A})gency -- the sense of participant ownership in workshop outcomes and research project;
\item(\textbf{C})hallenge -- the barrier of entry to, and likelihood of success in workshop methods; 
\item(\textbf{T})rust --  the confidence that participants have in the methods, the design process, and the researcher's visualization expertise; 
\item(\textbf{I})nterest -- the amount of attention, energy and engagement to workshop methods;
\item\textbf{+} -- ~other \emph{relevance} factors that can effect: the levels of engagement with \emph{data}, \emph{visualization} and the \emph{domain} in which collaborators are working.
\end{itemize}
The CACTI factors are neither independent, consistent, nor measurable. The extent to which they are fostered depends upon the context in which they are used, including various characteristics of the workshop group - often unknown in advance, though perhaps detectable by facilitators. And yet, maintaining appropriate levels of the factors likely helps workshops to inspire and engage participants while creating useful output and establishing lasting rapport among researchers and their collaborators.
\end{tcolorbox}