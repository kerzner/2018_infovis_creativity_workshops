\begin{table}
    \small
    \centering
    \begin{tabular}{|m{0.9\linewidth}|}
        \hline
        \textbf{Methods} \\
        \hline
        \textbf{\emph{Analogy Introduction}} --- a playful introduction to prime analogical thinking e.g., \emph{``if you were to describe yourself as an animal, what would you be?''}~[\ref{wor:eon}] \\
        \hline
        \textbf{\emph{Constraint Removal}} --- identifying constraints around current ideas, considering the removal of each constraint, and imagining  what would be possible~\cite{Jones2008}. \\ 
        \hline
        \textbf{\emph{Brainstorming}} --- unstructured ideation, \emph{``using the \emph{brain} to \emph{storm} a creative problem...through freewheeling generation of ideas''} \cite{Osborn1953}. \\
        \hline
        \textbf{\emph{Clustering}} --- grouping ideas into meaningful sets with descriptive titles\\
        \hline
        \textbf{\emph{Storyboarding}} --- creating a graphical depiction of a narrative ~\cite{Truong2006}\\
        \hline
        \textbf{\emph{Visual Improv.}} --- rapidly sketching concepts of increasingly complexity e.g., a line, a shape, a mountain, a pet, a mode of transportation, a dangerous mountain, a helpful pet, a friendly mode of transportation ... \\ 
        \hline
        \textbf{\emph{Visual Analogies}} --- displaying a curated set of visualization and encouraging ideation about how aspects of the visualizations may apply to a domain~\cite{Koh2011} \\
        \hline
        \textbf{\emph{Wishful Thinking}} --- prompted with a scenario, generating ideas in response to: What would you like to know? What would you like to see? What would you like to do? \\
        \hline
    \end{tabular}
    \caption{}
    \label{tab:methods}
\end{table}
