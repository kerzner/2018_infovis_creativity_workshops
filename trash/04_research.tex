\section{Research Process}
\label{sec:research-process}

This section summarizes our research and analysis process, conducted through an multiyear, cross-institution, international collaboration in which we have analyzed how and why to use creativity workshops in applied visualization research.

This paper’s contributions arise from reflection — the analysis of experiences to generate insights~\cite{Boud1985}. More specifically, we applied a methodology of critically reflective practice~\cite{Brookfield1998}, summarized as “synthesizing experience, reflection, self-awareness and critical thinking to modify or change approaches to practice”~\cite{Thompson2008}.

Our reflective methodology relied on a variety of methods to make sense of rich and descriptive workshop data which included workshop documentation and artifacts, participant feedback, experiential knowledge, and resulting research outputs. The specific methods included discussions, writing, {\it observation listing} and {\it observations-to-insights}~\cite{Kumar2012}. We also reviewed literature relevant to creativity and workshops, attempting to find guidance that applied directly to future visualization workshops. We repeatedly codified the outcomes of our analysis many times, sometimes individually and sometimes collaboratively, in both narrative and diagram form. Writing often identified shortcomings in our thinking, which we addressed in subsequent analysis to ultimately generate the workshop framework. 

More specifically, our analysis started with a seeming simply question about two workshops --- {\it what could we do better next time?} --- and its scope gradually expanded and evolved into the framework presented in this paper as we tried to understand why and how to use creativity workshops in applied visualization research.  Throughout this period, we searched existing literature on creativity and workshops~\cite{Biskjaer2017,CreativeEducationFoundation2015,DeBono1983,Gordon1961,Hamilton2016,Miller1989,Nickerson1999,Osborn1953,Sawyer2003,Sawyer2006,Shneiderman2005}, and this work provided import scaffolding for thinking about the process of using workshops. But it did not directly apply to visualization as, for example, it did not emphasize the role of data or visualization in workshops. 

Thus, our analysis focused on connecting the general process of workshops to applied visualization research --- connecting the domain-independent process to challenges commonly faced by visualization researchers. It also examined the workshop design, exploring how creativity methods could be adapted and adopted in the context of visualization. The result of our analysis is the framework presented in this paper, as well as a rich set of collected documentation captured throughout the two years. A detailed description of significant reflective events can be found in the Supplemental Material, along with an audit trail of documents that were produced throughout. 