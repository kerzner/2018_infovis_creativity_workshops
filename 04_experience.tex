\begin{table*}
    \small
    \centering
    \begin{tabular}{|lllllllll|}
        \hline
        \textbf{ID} & \textbf{Year} & \textbf{Domain} & \textbf{Summary} & \textbf{Workshops} & \textbf{Result} & \textbf{Prim.} & \textbf{Supp.} & \textbf{Ref.} \\
        \hline
        \customlabel{P1}{pro:edina} & 2009 & Cartography  & \emph{``“Reimagining the legend as an exploratory visualization interface”''} & 3 & Paper & JD & *  & \cite{Dykes2010} \\
        \customlabel{P2}{pro:eon} & 2012 & Smart Homes & Deliver insights into the role of smart homes and new business potential & 4 & Paper & SG & JD,SJ,* & \cite{Goodwin2013}\\
        \customlabel{P3}{pro:htva} & 2012 & Human terrain & \emph{``develop [visualization] techniques that are meaningful in HTA''} & 3 & Paper & JD & * & \cite{Walker2013}\\
        \customlabel{P4}{pro:graffinity} & 2015 & Neuroscience & Explore problem-driven multivariate graph visualization & 1 &  Paper & EK & MM, * & \cite{Kerzner2017} \\ 
        \customlabel{P5}{pro:cp} & 2015 & Constraint prog. & Design performance profiling methods for constraint programmers & 1 & Paper & SG & * & \cite{Goodwin2016}\\
        \customlabel{P6}{pro:lineage} & 2017 & Psychiatry & Support visual analysis of determining or associated factors of suicide & 1 & Paper & * & EK,* & \cite{Nobre2017}\\ 
        \customlabel{P7}{pro:updb} & 2017 & Genealogy  & Discover opportunities to support visual genealogy analysis & 1 &  ---  & * & EK,MM,* & \cite{Kerzner2017:utdb}\\
        \customlabel{P8}{pro:arbor} & 2017 & Biology & Support phylogenetic analysis with visualization software & 1 &  In-progress & * & EK,MM,*  & \cite{Lisle2017}\\
        \hline
    \end{tabular}
    \caption{Summary of the projects in which we have used \workshops: six resulted in publications [\ref{pro:edina} -- \ref{pro:lineage}], one did not result in active collaboration  [\ref{pro:updb}], and one is in progress [\ref{pro:arbor}]. We characterize our involvement in these projects as either the primary researcher or as supporting researchers. The * represents colleagues who were involved in each project but who are not coauthors of this paper.}
    \label{tab:projects}
\end{table*}
\begin{table}
    \small
    \centering
    \begin{tabular}{|lp{4cm}lll|}
        \hline
        \textbf{ID} & \textbf{Theme} & \textbf{Facil.} & \textbf{Partic.} & \textbf{Hrs} \\ \hline
        {\bf \ref{pro:edina}} & Explore possibilities for enhancing legends with visualizations & 1v & 3v / 5c & 6 \\
        %\customlabel{\ref{pro:edina}.D}{wor:edina:des} & Candidate solutions identified and considered in light of identified requirements & $\Circle$ & $\CIRCLE$ & $\CIRCLE$ & & Des. & 1v & 3v / 5c & 6 \\
        %\customlabel{\ref{pro:edina}.E}{wor:edina:eva} & Presentation and evaluation of deliverables & $\Circle$ & $\Circle$ & $\CIRCLE$ & & Eva. & 1v & 3v / 3c & 4 \\
        \hline
        {\bf \ref{pro:eon}} & Identify future opportunities for utilising smart home data/technologies & 2v / 1p & 0v / 5c & 6 \\
        %\customlabel{\ref{pro:eon}.D1}{wor:eon:des1} & Develop concepts from req. workshop in an agile approach & $\Circle$ & $\CIRCLE$ & $\Circle$ & & Des. & 2v & 6v / 0c & 4 \\
        %\customlabel{\ref{pro:eon}.D2}{wor:eon:des2} & Elicit feedback from prototypes and prioritize design improvements & $\CIRCLE$ & $\Circle$ & $\CIRCLE$ &  & Des. & 2v & 0v / 7c & 3 \\
        %\customlabel{\ref{pro:eon}.E}{wor:eon:eva} & Evaluate final prototypes & $\Circle$ & $\CIRCLE$ & $\CIRCLE$ & & Eva. & 2v & 0v / 5c & 3  \\
        \hline
        {\bf \ref{pro:htva}} & Identify novel visual approaches most suitable for HTA & 1v / 1p & 7v / 6c  & 9  \\
        %\customlabel{\ref{pro:htva}.D}{wor:htva:des} & To further establish requirements ... to acquire feedback on initial designs & $\CIRCLE$  & $\Circle$ & $\CIRCLE$ &  & Des. & 1v & 6v / 3c & 7 \\
        %\customlabel{\ref{pro:htva}.E}{wor:htva:eva} & Structured evaluation against scenarios & $\Circle$ & $\CIRCLE$ &  $\CIRCLE$ & & Eva. & 1v & 6v / 3c & 4  \\
        \hline
        {\bf \ref{pro:graffinity}} & Explore shared user needs for visualization in retinal connectomics & 4v & 0v / 9c & 7 \\
        \hline
        {\bf \ref{pro:cp}} & Identify analysis and vis. opportunities for improved profiling of cons. prog. & 2v / 1c & 0v / 10c & 7 \\
        \hline
        {\bf \ref{pro:lineage}} & Understand the main tasks of psychiatric researchers & 2v & 1v / 6c & 3  \\
        \hline
        {\bf \ref{pro:updb}} & Explore opportunities for a design study with genealogists & 1v & 3v / 7c  & 3 \\
        \hline
        {\bf \ref{pro:arbor}} & Explore opportunities for funded collaboration between vis. and bio. & 1v / 1c & 2v / 12c & 7x2 \\
        \hline
    \end{tabular}
    \caption{Summary of the \workshop used in each project. We describe workshops by their theme, a concise statement the topics explored. We characterize workshop stakeholders as facilitators or participants categorized by their affiliation as (v)isualization researchers, (c)ollaborators, or (p)rofessional facilitators. Our workshops included 5 -- 14 participants and ranged in length from half a day to 2 days.}
    \label{tab:workshops}
\end{table}

\section{Workshop Experience and Terminology}
\label{sec:experience}

To develop the \workshop framework proposed in this paper, we gathered researchers who used workshops on 3 continents over the past 10 years. Our collective experience includes 17 workshops in 10 contexts: 15 workshops in 8 applied collaborations, summarized in Table~\ref{tab:projects} and Table~\ref{tab:workshops}; and 2 participatory workshops at IEEE VIS that focused on creating visualizations for domain specialists~\cite{Rogers2016,Rogers2017}.

The ways in which we use workshops have evolved over 10 years. In three of our projects, we used a series of workshops to explore opportunities, develop and iterate on prototypes, and evaluate the resulting visualizations in collaborations with cartographers~\cite{Dykes2010}, energy analysts~\cite{Goodwin2013}, and defense analysts~\cite{Walker2013}. In three additional projects, we used a single workshop to jump-start applied collaborations with neuroscientists~\cite{Kerzner2017}, constraint programmers~\cite{Goodwin2016}, and psychiatrists~\cite{Nobre2017}.  Recently, we used two workshops to explore opportunities for funded collaboration with genealogists~\cite{Kerzner2017:utdb} and biologists~\cite{Lisle2017}.

In our meta-analysis, we focused on the workshops used in the early stages of applied work or as the first in a series of workshops. To describe these workshops, we developed the term \workshops because they aim to deliberately and explicitly foster creativity while exploring opportunities for applied visualization collaborations.

Focused on \workshops, our experience includes the eight workshops in Table~\ref{tab:workshops}. Since we analyzed more data than appeared in any resulting publications, including artifacts and experiential knowledge, we refer to workshops and their projects by identifiers, e.g., [\ref{pro:edina}] refers to our collaboration with cartographers. In projects where we used more than one workshop [\ref{pro:edina} -- \ref{pro:htva}], the identifier corresponds to the {\it first} workshop in the series, unless otherwise specified. 
% Requirements workshops typically focus on eliciting opportunities for visualization software from collaborators. In general, they support the {\it understand} and {\it ideate} design activities of the design activity framework~\cite{McKenna2014} or fulfill the {\it winnow}, {\it cast}, and {\it discover} stages of the design study methodology's nine-stage framework~\cite{Sedlmair2012}.

To describe our experience, we developed terminology for the role of researchers involved in each project. The {\bf primary researcher} is responsible for deciding to use a \workshop, executing it, and integrating its results into a collaboration. Alternatively, {\bf supporting researchers} provide guidance and support to the primary researcher. We have been involved with projects as both primary and supporting researchers (see Table~\ref{tab:projects}).  

We also adopt terminology to describe \workshops. Workshops are composed of {\bf methods} --- specific, repeatable and modular activities~\cite{Crotty1998}. The methods are designed around a {\bf theme} that identifies the workshop's central topic or purpose~\cite{Brooks-Harris1999}. The {\bf facilitators} plan and guide the workshop, and the {\bf participants} carry out the workshop methods. Typically the facilitators are visualization researchers and participants are domain collaborators, but, visualization researchers can participate [\ref{pro:edina},~\ref{pro:htva}], and collaborators can facilitate [\ref{pro:cp},~\ref{pro:arbor}]. We adopted and refined this vocabulary during our reflective analysis.