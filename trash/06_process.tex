\section{Workshop Process and Recommendations}
\label{sec:process}

We introduce a three-stage model that abstracts and simplifies the process of using creativity workshops in applied research. The model's stages correspond to the actions of designers before, during, and after a workshop. They are completed in a forward linear fashion and decisions cascade between stages as, for example, decisions made about the workshop participants will influence how the workshop is executed. 

The first stage is {\bf decide \& design}, in which we decide to use a workshop and decide on its purpose, participants and constraints. Based on these decisions, we design the workshop, selecting and tailoring creativity methods for the intended outcome. Often, the act of designing workshops forces us to analyze our understanding of the domain, and may influence the decisions about the workshop as we revisit assumptions about its purpose. This stage results in a flexible {\bf workshop plan} that outlines methods we intend to use in the workshop and resolves practical concerns, such as the duration, venue, materials, and participants.

The second stage is {\bf execute \& adapt}, in which we perform the workshop plan, adapting it to the emergent ideaspace and group creativity. This stage results in {\bf workshop output}, a set of rich and descriptive artifacts, participant feedback, and notes documenting the experience.

The third stage is {\bf analyze \& act}, in which we analyze workshop output, creating insights that we act on throughout the design process. Analysis and action are mutually influential as we learn through action and reanalyze workshop output with this new knowledge. This stage results in actionable knowledge integrated into the design process, for example, by creating prototypes inspired by ideas discussed in the workshop.

We use the process model to present a series of \reviseme{x} actionable recommendations that outline likely beneficial courses of action for future workshops. The process and recommendations are useful for researchers conducting applied projects who are interested in applying a creativity workshop.

\begin{table}
    \small
    \centering
    \begin{tabular}{|m{0.06\linewidth}|p{0.8\linewidth}|}
    \hline
    ID & Recommendation \\
    \hline
    \rec{rec:theme}{} & Articulate a mutually beneficial workshop theme. \\
    \rec{rec:participants}{} & Recruit diverse and creative participants. \\
    \rec{rec:diverse}{} & Recruit engaged and excited participants and facilitators. \\
    \rec{rec:vocab}{} & Evaluate domain knowledge before designing. \\
    \rec{rec:constraints}{} & Recognize workshop constraints to winnow the design space. \\
    \rec{rec:improvise}{} & Recognize tradeoffs of scripted vs improvised. \\
    \rec{rec:pilot}{} & Pilot workshops to test, evaluate, and improve methods. \\
    \hline
    \rec{rec:prepare}{} & Prepare for execution with facilitation resources. \\
    \rec{rec:materials}{} & Gather appropriate materials for workshop methods. \\
    \rec{rec:guidance}{} & Guide the workshop to foster creativity. \\
    \rec{rec:adapt}{} & Adapt facilitation to participant reactions. \\
    \rec{rec:artifacts}{} & Create artifacts --- anything not written will likely be lost. \\
    \hline
    \rec{rec:inspiring}{} & Expect rich and inspiring artifacts that warrant action. \\ 
    \rec{rec:time}{} & Allocate significant time for workshop analysis --- a day or more. \\
    \rec{rec:complement}{} & Complement workshops with traditional user-centered design methods.  \\
    \rec{rec:generative}{} & Use workshop outputs for divergent visualization design methods.  \\ 
    \rec{rec:evaluative}{} & Use workshop outputs for convergent visualization design methods.  \\
    \rec{rec:revisit}{} & Revisit workshop output and insights throughout the project. \\
    \hline
    \end{tabular}
    \caption{Summary of the workshop process recommendations.}
    \label{tab:recommendations}
\end{table}

\subsection{Before the Workshop: Decide \& Design}

We describe a series of decisions that can be used to determine if a workshop would be useful in specific projects. These decisions --- about the workshop theme, participants, facilitators, logistics, and constraints --- provide criteria for designing a workshop which involves selecting, tailoring, or inventing methods to use in it.

\subsubsection{Decide}

The process starts with a decision to use a workshop. This can be motivated by many factors, as workshops help to deliberately and explicitly stimulate creativity in a project [\ref{pro:eon}]; to sample problems faced by analysts in different organizations [\ref{pro:cp}]; to explore shared needs from diverse analysts [\ref{pro:graffinity},~\ref{pro:cp},~\ref{pro:lineage}]; to make use of limited meeting time with collaborators [\ref{pro:edina},~\ref{pro:htva},~\ref{pro:arbor}]; and to identify surrogate data if real data are not available [\ref{pro:htva}]. More generally, workshops can help to establish rapport with analysts and to rapidly characterize domain challenges while exploring specific analysis needs.

After deciding to use a workshop, articulating a central theme or goal will help to identify and recruit participants as well as design the workshop methods. The themes of our workshops have focused on specific analysis tasks, such as to explore specific domain challenges or visualization opportunities within an organization, such as \emph{``enhancing legends with visualizations''} [\ref{wor:edina}]. We have also broadly themed our workshops, such as \emph{``identify analysis and visualization opportunities for improved profiling of constraint programmers''} [\ref{wor:cp}]. While the theme of a workshop often evolves throughout the process of designing it, articulating the theme early can provide criteria for subsequent workshop decisions [\ref{rec:theme}].

The theme helps us to decide who should participate in the workshop as we recruit participants who can reasonably contribute interesting ideas that are relevant to the theme. We have typically recruited workshop participants within the project constraints --- such as in projects where we worked with specific individuals within a larger organization [\ref{wor:eon},~\ref{wor:htva},~\ref{wor:graffinity}]. We have also used online surveys of practitioners, students, and educated to identify potential participants based on their interesting or creative responses as well as desired interest to participate [\ref{wor:cp}]. Across all workshops, recruiting diverse and creative participants may contribute to the successful workshop execution and outcomes as they contribute experiential knowledge that enables exploration of a broader ideaspace [\ref{rec:diverse}].


%Generally, participants in our workshops tended to be frontline analysts, managers, and support staff. 
%Front line analysts are appropriate participants for workshops focused on specific analysis tasks --- \emph{``understand the main tasks of psychiatric researchers''} [\ref{wor:lineage}]. Diverse participants may contribute to more general themes, such as in the mix of practitioners, students and educators who participated in  
% When working with broader themes, we have recruited participants who can contribute diverse perspectives about the domain challenges. For example, we have used an online survey of practitioners, students, and educators to identify potential participants [\ref{wor:cp}] based on their interesting or creative responses and interest in continued participation. The survey had an added benefit of collecting relevant domain knowledge, that was used to tailor the workshop methods. When working within an organization, we have recruited diverse members of the research lab --- including graduate students, support staff, and professors . Yet, there have also been cases where we had little control over the participants, such as collaborators who identified participants as a workshop constraint [\ref{wor:eon},\ref{wor:htva}]. 


The participants will influence the workshop {\bf logistics}, such as the length and venue. Generally, workshops lasting one day (6 - 8 hours) provided appropriate time for creative thinking [\ref{wor:edina},~\ref{wor:eon},~\ref{wor:htva},~\ref{wor:graffinity},~\ref{wor:cp}], half day workshops can work, but may feel rushed and do not allow for sufficient incubation and iteration [\ref{wor:lineage},~\ref{wor:updb}], and two days [\ref{wor:arbor}], although productive, are a large commitment from collaborators. The length may influence the venue where the workshop will be executed. Creativity literature expounds the importance of neutral, well-lit venues~\cite{CreativeEducationFoundation2015,Isaksen2000}. While such venues can be successful [\ref{wor:eon},~\ref{wor:htva}], we have also had success hosting workshops in on-site conference rooms  [\ref{wor:graffinity},~\ref{wor:cp},~\ref{wor:lineage}]. The venue affordances, such as the room size and physical layout, are important factors in designing the workshop. 

The theme, logistics, and participants are important factors, but there are other constraints that should be decided to winnow the space of possible designers and increase the likelihood that the workshop designed will also be executed [\ref{rec:constraints}]. The set of relevant constraints varies between projects, but can include: the ability for collaborators to share data with researchers [\ref{pro:htva},~\ref{pro:lineage}], whether project stakeholders have to travel significant distances for meetings [\ref{pro:edina},~\ref{pro:arbor}], and the funding available for workshop materials. Listing such constraints will be useful for designing an appropriate workshop.

\subsubsection{Design}

Although the exact design process depends on the preferences and experiences of the facilitator, ranging from more scripted to more improvisational [\ref{rec:improvise}]. These preferences appear in our workshop experience, as we have improvised and adapted to the changing environment, for example, when participants took initiative and proposed a method that they would find helpful [\ref{wor:htva}]. On the other hand, other workshops relied on more formal planning and testing of methods before the workshop [\ref{wor:eon},~\ref{wor:graffinity},~\ref{wor:arbor}].

There are many ways to design useful workshops and the ideal way depends on the preference and experience of the designer. One approach to design a workshop is to outline a workshop plan by starting with time constraints---starting and ending times, as well as breaks for lunch and snacks. The time constraints become scaffolding which can be used to organize methods that will be used in the workshop while also following the workshop structure described in Sec.~\ref{sec:overview}. The example workshop, in Sec.~\ref{sec:overview}, illustrates this style of design as it uses divergent, active methods before lunch, and concludes with convergent methods near the end of the day.

There are a plethora of resources available for selecting methods to use in a workshop. Resources that we have found particularly useful include: McKenna et al.~\cite{McKenna2014}, who provide 100 exemplar methods relevant for visualization researchers, but these methods may need to be adapted to a workshop setting; Kumar~\cite{Kumar2012}, whom describes 101 product design methods useful in a business setting; Gray et al.~\cite{Macanufo2010}, who describe methods that encourage creative thinking and can be chained together into workshops. We have also drawn inspiration from Michalko~\cite{Michalko2006}, who describes creative-thinking techniques to approach and solve problems at home and in business; as well as Hohmann~\cite{Hohmann2007} who describe methods in the form of innovative games for improving workplace engagement, idea generation and communication. In addition, online descriptions of creativity methods may also be valuable~\cite{MyCotedInc2018,Maiden2018CreativeEngine}. The methods described in these resources should be used to inspire the creation of new and useful visualization creativity methods that incorporate data, visualization, analysis and automation as described in Sec.~\ref{sec:workshop-methods}. Knowing these resources and finding alternatives as they emerge, adopting methods in light of likely effects on \emph{CACTI} factors, and logging experiences are important components of creative visualization workshop design.

Methods should also be tailored to the specific workshop theme and project goals, through the use of appropriate prompts. Consider \emph{wishful thinking}, the second method in our example workshop (Fig.~\ref{fig:example-list}), where participants are prompted with a domain-specific scenario. We adjusted this scenario to the workshop theme as in a collaboration to explore long term goals for emerging technology~[\ref{wor:eon}], we asked participants to think about their \emph{``aspirations for the SmartHome programme...''}, which generated forward-thinking ideas about energy consumption, such as to better understand \emph{``the value of the data.''} In collaborations to explore current analysis needs, we asked  neuroscientists~[\ref{wor:graffinity}], \emph{``suppose you are analyzing a connectome...''} and constraint programmers~[\ref{wor:cp}], \emph{''your program does not execute as expected...''} Participant responses revealed shorter term goals, to \emph{``to understand neuron connectivity''} and to \emph{``explore the [solver] search space,''} respectively. 

Fine-tuning the workshop methods and prompts is an iterative process of expressing, testing, evaluating, and improving methods. Piloting workshop methods or the entire workshop is invaluable for this process [\ref{rec:pilot}]. We have used pilots to test how understandable are methods [\ref{wor:eon},~\ref{wor:graffinity}]; to evaluate whether method prompts create interesting results [\ref{wor:lineage},~\ref{wor:arbor}]; and to find errors in method prompts and materials [\ref{wor:eon},~\ref{wor:graffinity},~\ref{wor:lineage},~\ref{wor:arbor}]. Pilots can be run with proxy workshop participants, such as visualizations researchers~[\ref{wor:eon}] or domain collaborators~[\ref{wor:arbor}] --- providing an opportunity to improve our understanding of the domain challenges.  Importantly, piloting the workshop with all of the facilitators can influence its smooth execution. Even when reusing a known workshop structure tailored to a new domain or with a new facilitator team, a pilot would have allowed the team to be better prepared for their individual roles on the day as well as gain a better understanding of the methods and expected outcomes [\ref{wor:cp}].

\ek{Consider framing workshop as creativity support tools?}

\subsection{During the Workshop: Execute \& Adapt}

We describe important aspects of workshop execution, such as preparing for the workshop, efficiently guiding methods, and creating useful artifacts. We also describe strategies for adapting the workshop based on the continuous feedback from participants.

Having designed the workshop, we prepare for execution by reviewing the guidelines of effective workshop facilitation from the literature (e.g., \cite{Brooks-Harris1999,CreativeEducationFoundation2015,Hamilton2016,Macanufo2010,Stanfield2002}). This reveals common principles to keep in mind while executing the workshop, including: being energized, providing encouragement, demonstrating acceptance, using humor, and being punctual [\ref{rec:prepare}].  Gathering the correct materials is also important [\ref{rec:materials}] --- we have mistakenly bought post-it notes that are too big, causing participants to write more than one idea on a sheet and making it challenging to use methods that involve sorting or ranking ideas. Preparing the venue is also important, as the furniture arrangement should promote a feeling of co-ownership and encourage participation --- a semi-circle seating arrangement works well for this~\cite{Vosko1991}. A mistake in one of our workshops was to have the speaker using a podium, which implied a hierarchy between facilitators and participants, hindering communication~\cite{Rogers2016}.

\subsubsection{Execute}

While executing the workshop, we guide participants through the planned methods. Participants, as well as facilitators, should be focused on the workshop without distractions, such as leaving for a meeting or checking e-mail [\ref{rec:participants}]. In our experience, individuals in the workshop who are distracted by communication from beyond the workshop, or who are observing workshop methods, distract both the participant and the stakeholders, disrupting the smooth execution. [\ref{wor:graffinity},~\ref{wor:arbor}]. Expectations of focused thinking should be reinforced during the workshop opening (e.g., \emph{switch off all electronic devices}) and they should be clearly stated before the workshop.

While workshops enable participants to consider and explore a variety of ideas, ideas that are discussed but not written will likely be forgotten. Execution should focus on creating artifacts that capture the workshop ideaspace [\ref{rec:artifacts}]. In one case, audio recordings provided valuable information [\ref{wor:lineage}], but audio for longer workshops may not be useful as it requires tremendous time to transcribe and analyze after the workshop~\cite{Lloyd2011}. Recording may also hinder creativity as participants become self conscious\ek{SJ - do you have a citation for this?}. We make an effort to document all activities in the workshop, by note taking or through methods that create artifacts. The workshop team must know the expectations for note taking and pilot workshops will help with this. A pilot for [\ref{wor:cp}] for example, may have reduced the note taking pressure on the primary researcher during the day.

During execution the workshop team must guide participants through the methods, allowing for exploration but moving toward the common goal [\ref{rec:guidance}]. Conversations that deviate from the day's focus should be redirected, but this requires careful judgment to determine whether a conversation is likely to be fruitful and sensitivity about the creative atmosphere. When allowed to discuss freely, participants commented \emph{``we had a tendency to get distracted [during discussions]''}[\ref{wor:graffinity}]. Whereas more active guidance resulted in feedback: \emph{``we were guided and kept from going too far off track despite our tendencies to do so. This was very effective''} [\ref{wor:arbor}]. Yet, redirection can be jolting and can contradict some of the agreed guidelines (e.g., \emph{``all ideas are valid!''}). It may be beneficial to prepare participants for redirection with another guideline during the workshop opening: \emph{``facilitators may keep you on track gently, so please be sensitive to their guidance.''}

\subsubsection{Adapt}

As facilitators guide the workshop, they can interpret group dynamics to adapt to the changing situation [\ref{rec:adapt}]. If participants do not find a method helpful, they may propose their own as when analysts proposed walking through visualization analysis scenarios in place of a planned method [\ref{wor:htva}]. Facilitators should be prepared for flexibility, perhaps by having alternative methods planned or by being ready to improvise. It requires judgment to deviate from the plan, and the design considerations should be considered on-the-fly as the workshop adapts to participant responses.

The CACTI factors, from Sec.~\ref{sec:overview}, should be considered while adapting the workshop as facilitators respond to changing situations, such as a failing method (\emph{nobody feels like an animal this morning}; \emph{post-its don't stick}), a loss of interest (\emph{there is no energy}; \emph{the room is too hot}; \emph{we had a tough away day yesterday}) or a lack of agency (\emph{some participants dominate some tasks}). Given our collective experience, we offer some examples of methods in action that show how they have worked and failed in light of these 5 factors...

\subsection{After the Workshop: Analyze \& Act}

After the workshop we analyze the output, creating actionable knowledge that can influence the continued creative collaboration. For clarity, we discuss analysis and action separately, but the two are closely interrelated in our experience.

\subsubsection{Analyze}

Effective workshops generate rich and inspiring artifacts that can include hundreds of post-it notes, posters, sketches, and other items of documentation. Making sense of this output is labor intensive, often requiring more time than the workshop itself. Thus, it is important to allocate time for analysis, particularly within a day or so of the workshop, so that ideas are fresh in memory [\ref{rec:time}].

Typically, the primary researcher analyzes the output as they are using it to shape an ongoing design conversation with their collaborators. Clearly identifying the primary researcher before this stage is important as they decide how to analyze the workshop output and what to do with the results of that analysis. In our failed project [\ref{wor:updb}], we ran a workshop without clearly identifying the primary researcher, and workshop output went unused.

In our experience, we have analyzed workshop output by typing or photographing artifacts into documents or spreadsheets, allowing  us to become familiar with all ideas in the artifacts. This also enables sharing the output to enlist diverse stakeholders --- such as collaborators or other workshop team members --- in making sense of the results and clarifying ambiguous requirements. This is particularly important in domains with complex vocabulary.

The specific analysis methods will depend on the form of the artifacts which is directly influenced by workshop methods. In most cases [\ref{wor:eon},~\ref{wor:graffinity},~\ref{wor:cp},~\ref{wor:lineage},~\ref{wor:updb}],  we used qualitative analysis methods -- open coding, mindmapping, and other less formal processes -- to group workshop artifacts into common themes or tasks. We often ranked these themes and tasks by various criteria, including, novelty, ease of development, potential impact on the domain, and relevance to the collaboration.  In other cases~[\ref{wor:edina},~\ref{wor:htva}], workshop methods generated specific requirements, tasks, or scenarios that could be editing for clarity and directly integrated into the design process. Quantitative analysis methods should be approached with caution as the frequency of an idea provides little information about its novelty, usefulness, or potential impact. The insights gleaned from analysis will influence many aspects of the remaining design process. 

\subsubsection{Act}

The results of analysis have influenced our understanding of the domain challenges and specific analysis tasks. We have used workshop results to scope traditional  user-centered design methods, such as interviews and contextual inquiry [\ref{rec:complement}]. For example, a common theme of output from our neuroscience workshop was to \emph{``analyze multi-hop relationships''} [\ref{wor:graffinity}]. Using this theme, we focused interviews on the challenges of analyzing connectivity, revealing low-level tasks that inspired subsequent prototypes.

The results of the workshop can be used to create prototypes of varying fidelity, from sketches to functioning software. For example, we have used the workshop output in parallel prototyping [\ref{wor:graffinity},\ref{wor:cp},], as well as to decide on features for in-development software tools~[\ref{wor:lineage}], as one of our collaborators who used the workshop told us \emph{``I personally got a much better understanding of what they were trying to do and what information they needed to do it ... which ultimately guided our design decisions.''} In other cases [\ref{wor:edina},~\ref{wor:eon},~\ref{wor:htva}], we have used the workshop output as input to additional workshops focused on rapidly exploring the possibilities for visualization design. 

These activities may adapt existing software to newly discovered analysis needs or explore entirely new visualization techniques as in our neuroscience project~\ref{wor:graffinity}, where the outputs inspired plugins for existing tools that we iteratively developed into a novel visualization technique. In all of these cases, our actions can be considered divergent --- expanding space of possible visualization designs currently being considered [\ref{rec:generative}]. The results can be used in convergent design methods --- contracting the space of possible visualization designs [\ref{rec:evaluative}]. The workshop output can involve design considerations, such as reaching \emph{``everything in three clicks''} [\ref{wor:eon}] and providing \emph{``access [to] underlying database keys''} [\ref{wor:graffinity}] from visualizations. These criteria can be used to winnow the space of possibilities, for example, to evaluate, focus, and refine designs and prototypes. 

We emphasize that analyzing the output and acting on the results of analysis occur iteratively and that workshop output should be revisited throughout the project [\ref{rec:revisit}]. Workshop artifacts can provide valuable evidence about the contributions of applied work as they can document that visualization systems fulfill real analysis needs. They can also be used to document the evolution of ideas that occurs throughout design studies.