\section{Before the Workshop: Define \& Design}
\label{sec:before}

Creating an effective \workshop is a design problem: there is no single correct workshop, the ideal workshop depends on its intended outcomes, and the space of possible workshops is practically infinite. Accordingly, workshop design is an iterative process of defining a goal, testing solutions, evaluating their effectiveness, and improving ideas. The framework we have developed here is part of this process. In this section, we introduce four guidelines --- summarized in paragraph-level headings --- for workshop design.

\paragraph{Define the theme.} Just as design starts with defining a problem, creating a \workshop starts with defining its purpose, typically by articulating a concise theme.
An effective theme piques \interest in the workshop through a clear indication of the \topic. It encourages a \mindset of mutual learning among stakeholders. It also focuses on opportunities that exhibit the appropriate {\it task clarity} and {\it information location} of the design study methodology~\cite{Sedlmair2012}. Examples from our work emphasize visualization opportunities (e.g., {\it ``enhancing legends with visualizations''} [\ref{pro:edina}]), domain challenges (e.g., {\it ``identify analysis and visualization opportunities for improved profiling of constraint programs''} [\ref{pro:cp}]), or broader areas of mutual interest (e.g., {\it``explore opportunities for a funded collaboration with phylogenetic analysts"}~[\ref{pro:arbor}]).

Although we can improve the theme as our understanding of the domain evolves, posing a theme early can ground the design process and identify promising participants.

\paragraph{Recruit diverse and creative participants.} \label{par:participants} We recruit participants who have relevant knowledge and diverse perspectives about the \topic. We also consider their openness to \challenge and potential \collegiality.

Examples of effective participants include a mix of frontline analysts, management, and support staff [\ref{pro:graffinity}]; practitioners, teachers, and students [\ref{pro:cp}]; or junior and senior analysts [\ref{pro:lineage}]. We recommend that participants attend the workshop in person because remote participation proved distracting in one workshop~[\ref{pro:arbor}]. Recruiting {\it fellow-tool builders}~\cite{Sedlmair2012} as participants should be approached with caution because their perspectives may distract from the \topic~--- this happened in our workshop that did not result in active collaboration [\ref{pro:updb}].

\paragraph{Design within constraints.} Identifying constraints can help winnow the possibilities for the workshop. Based on our experience, the following questions are particularly useful for workshop design:

\begin{itemize}[nolistsep,noitemsep]

\item Who will use the workshop results? Identifying the primary researcher early in the process is important because he or she will be responsible for the workshop and ultimately use its results. In a workshop where we did not clearly identify the primary researcher, the results went unused [\ref{pro:updb}].

\item How many participants will be in the workshop? We typically recruit 5 to 15 participants --- a majority domain collaborators, but sometimes designers and researchers [\ref{pro:edina},~\ref{pro:htva},~\ref{pro:lineage}~--~\ref{pro:arbor}].

\item Who will help to facilitate the workshop? We have facilitated our workshops as the primary researcher, with the assistance of supporting researchers or professional workshop facilitators. Domain collaborators can also be effective facilitators, especially if the domain vocabulary is complex and time is limited [\ref{pro:cp},~\ref{pro:arbor}].

\item How long will the workshop be? Although we have run workshops that range from half a day [\ref{pro:lineage},~\ref{pro:updb}] to two days [\ref{pro:arbor}], these extremes either feel rushed or require significant commitment from collaborators. We recommend that an effective workshop lasts about one working day.

\item Where will the workshop be run? Three factors are particularly important for determining the workshop venue: a mutually convenient location, a high quality projector for visualization examples, and ample space to complete the methods. We have had success with workshops at offsite locations [\ref{pro:eon},~\ref{pro:htva}], our workplaces, and our collaborators' workplaces [\ref{pro:graffinity}~--~\ref{pro:lineage}].

\item What are additional workshop constraints? Examples include the inability of collaborators to share sensitive data [\ref{pro:htva},~\ref{pro:lineage}] and the available funding.

\end{itemize} 

\paragraph{Pilot the methods and materials.} Piloting methods can ensure that the workshop will generate ideas relevant to the \topic while maintaining appropriate levels of \interest and \challenge. We have piloted methods to evaluate how understandable they are [\ref{pro:eon},~\ref{pro:graffinity}], to test whether they create results that can be used to advance visualization design methodologies [\ref{pro:lineage},~\ref{pro:arbor}], to find mistakes in method prompts [\ref{pro:eon},~\ref{pro:graffinity},~\ref{pro:lineage},~\ref{pro:arbor}], and to ensure that the materials are effective --- e.g., \stickyNotes are the correct size and visualizations are readable on the projector.

It is also useful to pilot workshops with proxy participants, such as researchers~[\ref{pro:graffinity}] or collaborators~[\ref{pro:arbor}]. Feedback from collaborators during pilots has helped us revise the theme, identify promising participants, and refine the workshop methods.
