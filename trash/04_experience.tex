% \section{Project and Workshop Experience}
% \label{sec:experience}

% We analyzed 8 visualization projects that used 15 creativity workshops, summarized in Tab.~\ref{tab:projects} and Tab.~\ref{tab:workshops} respectively, as well as 2 participatory and creative workshops with a variety of domain specialists at the World's leading visualization conference --- IEEE Vis~\cite{Rogers2016,Rogers2017}. As we analyzed more data than appeared in the resulting publications, including workshop artifacts and experiential knowledge, we refer to projects and workshops by unique identifiers throughout this paper, e.g.,~[\ref{pro:edina}] and [\ref{wor:edina}]. This section describes the projects in which we have used workshops as well as details about workshops, such as their intended result, duration, and number of participants.

% \subsection{Projects}

% The projects in which we have used workshops were conducted over the past 10 years. They span 8 distinct domains, including geographic information systems [\ref{pro:edina}], smart homes [\ref{pro:eon}], the life sciences [\ref{pro:graffinity},~\ref{pro:lineage}~--~\ref{pro:arbor}], and constraint programming [\ref{pro:cp}]. Their goals ranged from documenting and exploring the potential of visualization within a domain [\ref{pro:edina}~--~\ref{pro:htva}], to creating tools that support existing analysis needs [\ref{pro:graffinity}~--~\ref{pro:lineage}], to exploring the possibilities for funded collaboration [\ref{pro:updb},~\ref{pro:arbor}]. A majority of the projects resulted in publications in the visualization research literature [\ref{pro:edina}~--~\ref{pro:lineage}], one project resulted in a funding proposal [\ref{pro:arbor}], and one project we consider to be a failure as it did not result in active collaboration [\ref{pro:updb}]. Furthermore, the projects were completed on three continents, conducted by researchers at City, University of London [\ref{pro:edina}~--~\ref{pro:htva}], the University of Utah [\ref{pro:graffinity},~\ref{pro:lineage}~--~\ref{pro:arbor}], and Monash University [\ref{pro:cp}]. The diversity of our projects, in terms of their location, domain collaborators, and outcomes provides evidence that creativity workshops are a valuable method for visualization researchers. It supports our claims of validity and contributes to the transferability of the framework.

% We classify our involvement in each project as either a primary or supporting researcher. The {\bf primary researcher} is responsible for deciding to use a workshop, executing the workshop, and integrating the workshop results into a collaboration through analysis and action. Alternatively, the {\bf supporting researchers} may assist in the workshop process and provide guidance to the primary researcher. We have analyzed experiences as primary researchers [\ref{pro:edina}~--~\ref{pro:cp}] and as supporting researchers [\ref{pro:lineage}~-- ~\ref{pro:updb}], contributing diverse perspectives to the framework.

% \subsection{Workshops}

% We describe workshops in terms of measurable characteristics, such as their duration. A majority of our workshops were about one working day in length [\ref{wor:edina}~--~\ref{wor:cp}], with other workshops ranging from a few hours [\ref{wor:lineage},~\ref{wor:updb}] to a few days [\ref{wor:arbor}]. We can also describe workshops in terms of the stakeholders involved as {\bf facilitators}, who guide and document the workshop execution, as well as the number {\bf participants}, who actually carry out the workshop methods. Our workshops typically included 1 -- 4 facilitators guiding 5 -- 17 participants through structured creativity methods. The facilitators were visualization researchers [\ref{wor:edina},~\ref{wor:graffinity},~\ref{wor:lineage},~\ref{wor:updb}] assisted by professional facilitators [\ref{wor:eon},~\ref{wor:htva}], or domain collaborators [\ref{wor:cp},~\ref{wor:arbor}]. Participants include analysts, managers, and support staff. The ratio of researchers to collaborators depends on the workshop's intended outcomes.

% We characterize the workshops in our experience by their intended outcomes, abstracting and simplifying their role in the design process. Specifically, we retrospectively categorize workshops on how they fulfill {\it design activities} from the design activity framework~\cite{McKenna2014}, as shown in Tab.~\ref{tab:workshops}. Reinforcing the terminology of Goodwin et al.~\cite{Goodwin2013}, we recognize three broad workshop focuses: {\bf requirements workshops} generate an early understanding of user needs and explore how visualization could be used in a domain, often before significant efforts to create or develop prototypes [\ref{wor:edina}~--~\ref{wor:arbor}]; {\bf design workshops} either generate design ideas to guide development~[\ref{wor:eon:des1}], or engage collaborators to evaluate designs and prototypes~[\ref{wor:edina:des},~\ref{wor:eon:des2},~\ref{wor:htva:des}]; and {\bf evaluation workshops} present and evaluate final prototypes, often to conclude a project [\ref{wor:edina:eva}~--~\ref{wor:htva:eva}].

% Granted: characterizing workshops by their role in the design process is imperfect because design is a messy, iterative process and our actions often influence it in unpredictable ways. Furthermore, the boundaries between workshop focuses are nebulous, and, to some extent, all of our workshops could be considered requirements workshops because applied visualization research is about understanding and exploring new uses of visualization. Nevertheless, the workshop focus provides terminology to identify similarities between workshops that have the same intended result. Requirements workshops, for example, encourage wide ranging discussion of possibilities for visualization within a domain. Design and evaluation workshops are more narrowly focused around prototypes and the application of techniques to address and identify usage scenarios. The workshop focuses are also related to the time remaining for collaboration as requirements workshops can explore a variety of ideas early in the project, design workshops gather feedback to guide iterative development, and evaluation workshops have a more summative role in concluding projects, delivering outputs and presenting and evaluating prototypes of varying fidelity. 

% {\it We developed the framework in this paper to understand how and why to use creativity requirements workshops in the early formative stages of applied research projects.} We scope this paper on requirements workshops because it is the focus which we consider to be the most valuable as creativity requirements workshops offer an alternative to the traditional time consuming process of discussions, interviews, and contextual inquiry~\cite{Sedlmair2012}. Furthermore, the subsequent design process is likely to be creative if linked to a preceding creative requirements workshop~\cite{Goodwin2013}. 

%This can be motivated by many factors, as workshops help to deliberately and explicitly stimulate creativity in a project [\ref{pro:eon}]; to sample problems faced by analysts in different organizations [\ref{pro:cp}]; to explore shared needs from diverse analysts [\ref{pro:graffinity}~--~\ref{pro:lineage}]; to make use of limited meeting time with collaborators [\ref{pro:edina},~\ref{pro:htva},~\ref{pro:arbor}]; and to identify surrogate data if real data are not available [\ref{pro:htva}].