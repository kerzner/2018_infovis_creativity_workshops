\section{During The Workshop: Execute \& Adapt}
\label{sec:during}

Continuing the \workshop process model shown in Fig.~\ref{fig:framework-overview}, we execute the workshop plan. This section proposes five guidelines for workshop execution.

\paragraph{Prepare to execute.} We prepare for the workshop in three ways: resolving details, reviewing how to facilitate effectively, and checking the venue. We encourage researchers to prepare for future workshops in the same ways.

We prepare by resolving many details, such as inviting participants, reserving the venue, ordering snacks for breaks, making arrangements for lunch, etc. Brooks-Harris and Stock-Ward~\cite{Brooks-Harris1999} summarize many practical details that should be considered in preparing for execution. Our additional advice is to promote the visualization mindset in workshop preparation and execution.

We prepare by reviewing principles of effective facilitation, such as acting professionally, demonstrating acceptance, providing encouragement, and using humor~\cite{CreativeEducationFoundation2015,Brooks-Harris1999,Gray2010,Hamilton2016,Stanfield2002}. We also assess our knowledge of the domain because, as facilitators, we will need to lead discussions. Effectively leading discussions can increase \collegiality and \trust between stakeholders as participants can feel that their ideas are valued and understood. In cases where we lacked domain knowledge, we recruited collaborators to serve as facilitators [\ref{pro:cp},~\ref{pro:arbor}]. 

We also prepare by checking the venue for necessary supplies, such as a high quality projector, an Internet connection (if needed), and ample space for group activity. Within the venue, we arrange the furniture to promote a feeling of co-ownership and to encourage \agency~--- a semi-circle seating arrangement works well for this~\cite{Vosko1991}. A mistake in one of our workshops was to have a facilitator using a podium, which implied a hierarchy between facilitators and participants, hindering \collegiality~\cite{Rogers2016}.

\paragraph{Limit distractions.} Workshops provide a time to step away from normal responsibilities and to focus on the \topic. Accordingly, participants and facilitators should be focused on the workshop without distractions, such as leaving for a meeting. 

Communicating with people outside of the workshop --- e.g., through e-mail --- commonly distracts participants and facilitators. It should be discouraged in the workshop opening (e.g., \emph{switch off all electronic devices}). Principles in the workshop opening, however, should be justified to participants. Also, facilitators should lead by example at the risk of eroding \trust and \collegiality.

% \paragraph{Get out of the way.} After the workshop opening establishes a creative atmosphere and fosters engagement, participants commonly take initiative in completing the workshop methods. Hence, we use the word {\it facilitator} to describe the individuals guiding the workshop because their role is to {\it facilitate} the exploration of ideas as opposed to {\it lead} or {\it command} the participants. To an extent, facilitating a workshop is like conducting an interview because we should stay quiet and try to keep the participants talking or generating ideas. 

\paragraph{Guide gently.} While starting execution, the workshop opening can establish an atmosphere in which participants take initiative in completing methods. It is, however, sometimes necessary to redirect the participants in order to stay focused on the \topic. Conversations that deviate from the workshop theme should be redirected. In one workshop [\ref{pro:graffinity}], participants were allowed to discuss ideas more freely, and they reported in feedback that, {\it ``We had a tendency to get distracted [during discussions].''} In a later workshop [\ref{pro:arbor}], we more confidently guided discussions, and participants reported  \emph{\it ``We were guided and kept from going too far off track \ldots this was very effective.''}

However, guiding participants requires judgment to determine whether a conversation is likely to be fruitful. It also requires us to be sensitive to the \tactics~--- e.g., how would redirecting this conversation influence \collegiality or \agency? Redirection can be jolting and can contradict some of the guidelines (e.g., \emph{all ideas are valid}). We can prepare participants for redirection with another guideline during the workshop opening: \emph{Facilitators may keep you on track gently, so please be sensitive to their guidance.} 

\paragraph{Be flexible.} As we guide participants to stay on topic, it is important to be flexible in facilitation. For example, we may spend more time than initially planned on fruitful methods or cut short methods that bore participants. 

Following this guideline can also blur the distinction between participants and facilitators. In one workshop [\ref{pro:htva}], participants proposed a method that was more useful than what was planned. Thus, they became facilitators for this part of the workshop, which reinforced \agency and maintained the \interest of all stakeholders in the project. In the future, we may explore ways to plan this type of interaction, perhaps encouraging participants to create their own methods. 

\paragraph{Adapt tactically.} As we guide the workshop, we interpret group dynamics and adapt methods to the changing situation. We can be forced to adapt for many reasons, such as a failing method (\emph{nobody feels like an animal this morning}; \emph{sticky notes don't stick}), a loss of \interest (\emph{there is no energy}; \emph{the room is too hot}; \emph{we had a tough away day yesterday}); a lack of \agency (\emph{some participants dominate some tasks}); or an equipment failure ({\it projector does not work}; {\it no WiFi connection to present online demos}~[\ref{pro:eon}]). Designing the workshop with alternative methods in mind --- perhaps with varying degrees of \challenge~--- can ensure that workshop time is used effectively.

\paragraph{Record ideas collectively.} Remember: conversations are ephemeral and anything not written down will likely be forgotten. We therefore encourage facilitators and participants to document ideas with context for later analysis. Selecting methods to create physical artifacts can help with recording ideas. As described in Sec.~\ref{sec:design}, externalizing ideas on \stickyNotes and structured prompts has been effective in our workshops and addresses the visualization mindset. 

We are uncertain about the use of audio recording to capture workshop ideas. Although it can be useful for shorter workshops [\ref{pro:lineage}], it can require tremendous time to transcribe before analysis~\cite{Lloyd2011}. Also, recording audio effectively can be challenging as participants move around during the workshop.

It can be useful to ensure that facilitators know that they are expected to help document ideas. A pilot workshop can help with this. In at least one of our projects [\ref{pro:cp}], a pilot workshop may have reduced the note taking pressure on the primary researcher by setting clear expectations that all facilitators should help take notes.


% \paragraph{Assemble alternatives.}If participants do not find a method helpful, they may propose their own as when analysts proposed walking through visualization analysis scenarios in place of a planned method [\ref{pro:htva}]. Facilitators should be prepared for flexibility, perhaps by having alternative methods planned or by being ready to improvise. It requires judgment to deviate from the plan, and the design considerations should be considered on-the-fly as the workshop adapts to participant responses.

