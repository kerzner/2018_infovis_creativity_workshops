\section{Conclusion and Future Work} 
\label{sec:conclusion}

This paper contributes a framework for using workshops in the early stages of applied visualization research. The framework consists of two models for \workshops~--- a process model and a workshop structure. The framework also includes \numberOfGuidelines actionable guidelines for future workshops and a validated example workshop. 

We support the framework with Supplemental Material that includes extended details about the example workshop, \numberOfExampleMethods additional example workshop methods, \numberOfPitfalls pitfalls to avoid in future workshops, and an analysis timeline and audit trail documenting how we developed the framework during a 2-year reflective collaboration. We hope that this framework inspires others to use and report on \workshops in applied visualization research.

Further thinking on the framework reveals opportunities for developing \workshop \methods that emphasize the visualization \mindset. For example, inspired by the Dear Data project~\cite{Lupi2016}, we could ask participants to create graphics that reveal something about their daily life in the week before the workshop. The Dear Data Postcard Kit~\cite{Lupi2017} offers guidance and materials for creating data visualizations about personal experiences, which could be adopted in \workshops.

We also hope to better understand the role of data in \workshops. Visualization methodologies stress the importance of using real data early in collaborative projects~\cite{Lloyd2011,Sedlmair2012}. Our workshops, however, have focused participants on their perceptions of data rather than using real data because working with data is time consuming and unpredictable. In some projects, we incorporated data into the design process by using a series of workshops spaced over weeks or months, providing time for developers to design prototypes between workshops [\ref{pro:edina} -- \ref{pro:htva}]. This development between workshops was expensive in terms of time and effort. But time moves on, and we may be able to reliably use data in workshops with new technologies and techniques, e.g., visualization design tools~\cite{Wongsuphasawat2016}, declarative visualization languages~\cite{Satyanarayan2017}, constructive visualization~\cite{Huron2014}, and sketching~\cite{Walny2015}.

Additionally, in this paper we focused on workshops to elicit visualization opportunities in the early stages of applied work. Exploring how the framework could be influenced by and extended for workshops that correspond to other stages of applied work --- including the creation and analysis of prototypes, the exploration of data, or in the deployment, training and use of completed systems --- may open up opportunities for using creativity in visualization design and research.