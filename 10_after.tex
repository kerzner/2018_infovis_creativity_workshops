\section{After the Workshop: Analyze \& Act}
\label{sec:after}

After the \workshop, we analyze its output and use the results of that analysis to influence the on-going collaboration. Here, we describe five guidelines for this analysis and action. 

\paragraph{Allocate time for analysis --- soon.} Effective \workshops generate rich and inspiring artifacts that can include hundreds of \stickyNotes, posters, sketches, and other documents. The exact output depends on the methods used in the workshop. Piloting methods can help prepare researchers for the analysis. Regardless, making sense of this output is labor intensive, often requiring more time than the workshop itself. Thus, it is important that we allocate time for analysis, particularly within a day of the workshop, so that we can analyze the workshop output while the experiences are still fresh in our memory.

\paragraph{Create a corpus.} We usually start analysis by creating a digital corpus of the \workshop output. We type or photograph the artifacts, organizing ideas into digital documents or spreadsheets. Through this process, we become familiar with key ideas contained in the artifacts. The corpus also preserves and organizes the artifacts, potentially allowing us to enlist diverse stakeholders --- such as facilitators and collaborators --- in analysis [\ref{pro:graffinity}]. This can help in clarifying ambiguous ideas or adding context to seemingly incomplete ideas.

\paragraph{Analyze with an open mind.} Because the ideas in the workshop output will vary among projects, there are many ways to analyze this corpus of artifacts. We have used qualitative analysis methods --- open coding, mindmapping, and other less formal processes --- to group artifacts into common themes or tasks [\ref{pro:eon},~\ref{pro:graffinity}~--~\ref{pro:updb}]. Quantitative analysis methods should be approached with caution as the frequency of an idea provides little information about its potential importance.

We have ranked the themes and tasks that we discovered in analysis according to various criteria, including novelty, ease of development, potential impact on the domain, and relevance to the project [\ref{pro:eon},~\ref{pro:graffinity}--~\ref{pro:lineage}]. In other cases~[\ref{pro:edina},~\ref{pro:htva}], workshop methods generated specific requirements, tasks, or scenarios that could be edited for clarity and directly integrated into the design process. 

We encourage that analysis be approached with an open mind because of the many ways to make sense of the workshop data, including some approaches that we may not yet have considered.

\paragraph{Embrace results in the visualization design process.} Similarly, \workshop results can be integrated into visualization methodologies and processes in many ways. We have, for example, run additional workshops that explored the possibilities for visualization designs [\ref{pro:edina},~\ref{pro:eon}]. We have applied traditional user-centered design methods, such as interviews and contextual inquiry, to better understand collaborators' tasks that emerged from the workshop~[\ref{pro:graffinity}]. We have created prototypes of varying fidelity, from sketches to functioning software [\ref{pro:graffinity}~--~\ref{pro:lineage}], and we have identified key aims in proposals for funded collaboration~[\ref{pro:arbor}]. 

In all of these cases, our actions were based on the reasons why we ran the workshops, and the workshop results profoundly influenced the direction of our collaboration. For example, in our collaboration with neuroscientists~[\ref{pro:graffinity}], the workshop helped us focus on graph connectivity, a topic that we were able to explore with technology probes and prototypes of increasing fidelity, ultimately resulting in new visualization tools and techniques.

\paragraph{Revisit, reflect, and report on the workshop.} The \workshop output is a trove of information that can be revisited throughout (and even beyond) the project. It can be used to document how ideas evolve throughout applied collaborations. It can also be used to evaluate and validate design decisions by demonstrating that any resulting software fulfills analysis needs that were identified in the workshop data~[\ref{pro:edina}~--~\ref{pro:lineage}]. Revisiting workshop output repeatedly throughout a project can continually inspire new ideas.

In our experience creating this paper, revisiting output from our own workshops allowed us to analyze how and why to use \workshops.  We encourage researchers to reflect and report on their experiences using \workshops, the ways in which workshops influence collaborations, and ideas for future workshops. We hope that this framework provides a starting point for research into these topics.