\section{Discussion}
\label{sec:discussion}

\ek{To discuss: the next two sections are a mess of scoping and hedging our contributions, identifying future research, and reflecting on our research process. Please think about how these could be reorganized.}

The contributions of this paper result from a methodology of critically reflective practice as we tried to discover, explore, and articulate our collective knowledge about the use of creativity workshops in applied visualization research. Throughout this process, we analyzed workshops using the four metaphorical lenses described by Brookfield~\cite{Brookfield1998}: our autobiographical lens, the lens of our peers (collaborators), the lens of existing theory, and the lens of our learners (readers). We used a variety of methods to perform this analysis, including discussions, writing, observations-to-insights and insight sorting~\cite{Kumar2012}, all of which are documented in our supplemental material. Although the resulting framework is a subjective interpretation of our experience, it is grounded and validated by similarities with creativity and workshop literature. Moreover, we believe that the framework is valuable because it is subjective --- it articulates the knowledge we have gained from running 15 workshops in 8 projects across 3 continents over 10 years.

Although our workshops span a variety of domains, we recognize that our experiences all look like standard design studies as visualization researchers working closely with domain collaborators. Analysts have been intrinsically motivated to work with us and excited to learn about visualization to better understand their data. While this is a common in design studies, it is also a potential limitation of our framework as we have not considered workshops with casual users, who may appreciate visualizations for their aesthetic or playful qualities~\cite{McCurdy2016}. We suspect that aspects of our framework may be relevant to casual users, but this is an interesting area for future work.

Overall, we consider this framework as a creativity support tool~\cite{Shneiderman2005} that encourages the \emph{creative} use of visualization creativity workshops. We see the framework as a starting point for exploring the space of possible workshops and their uses. For example, we have avoided strict guidelines about the duration and size of workshops to embrace diverse possibilities. While writing this paper, we worked with a separate visualization researcher who applied a workshop with 50 participants that was  beyond the scope of what we considered, but still valuable. We have also avoided discussing the practical considerations of workshop --- logistics, facilitation strategies, and time management --- of workshops because they are not specific to visualization. While many resources address considerations, three particularly useful resources are book by Brooks-Harris and Stock-Ward~\cite{Brooks-Harris1999}, Hamilton~\cite{Hamilton2016}, and Gray et al.~\cite{Macanufo2010}.

In our collaboration, we stumbled on many open questions about creativity workshops that are potentially interesting for future research. We would like to explore how to articulate and analyze the effect of particular methods and their use in particular workshop contexts on the emergent creativity that occurs in workshops. This could be useful for developing guidelines to adjust workshops on-the-fly, for example, by adapting methods if levels of engagement among participant are too low or too high, methods take more or less time than anticipated or the artifacts being produced are not useful. We are also interested in understanding how workshop methods can be used to explore the role of automation or the information location vs task clarity axes of the design study methodology~\cite{Sedlmair2012}. For example, this could be done by explicitly asking about automation in methods like Wishful Thinking---\emph{``what would you like the computer to know/see/do?''}. \jd{@EK: As I said earlier, I think we probably do do this implicitly anyway, as part of `analysis' -- so I would wrap this into the earlier writing and indicate that doing this explicitly relates our process to an existing process model}. 

%Although our analysis focused on requirements workshops, we would like to better understand how creativity workshops can be used for other design activities, including workshops to build and evaluate prototypes. \ek{JD, SG: can you speculate on how accurate or useful the framework would be for other types of workshops, e.g., [\ref{wor:edina:des},~\ref{wor:edina:eva},~\ref{wor:eon:des1},~\ref{wor:eon:des2},~\ref{wor:htva:des},~\ref{wor:htva:eva}]?} 
% \jd{We have to chat about this -- what we all mean by `prototypes' and how this fits in to DSM and DAF. I think we have to have this chat because I suspect that we have different ideas about prototypes. EK might even mean \emph{deployed} systems here, and even if he doesn't ... why not try to understand how creativity workshops can be used to evaluate deployed systems through the latter stages of (for example) a MILC}. 

% \ek{JD - I am using this definition of prototyping~\cite{Martin2012}: \emph{``tangible creation of artifacts at various levels of resolution, for development and testing of ideas within design.''} Here, I'm trying to ask about our experience using workshops in projects in which we have put in varying amounts of time into development. For example, these two workshops --- [\ref{wor:eon}] and [\ref{wor:eon:eva}] --- were different, in part, because in the latter one you had built prototypes and were using them to explore the design space. How do you think this framework (which was invented to describe [\ref{wor:eon}]) would apply to [\ref{wor:eon:eva}]?} 

% \ek{Absolutely, we should speculate about how workshops could be used to evaluate deployed systems as in MILCS, although we don't have much experience with that as shown in the deploy column of Tab.~\ref{tab:workshops}. There's a potential symmetry here about the design cycle: a workshop to evaluate a deployed system is, to some extent, a requirements workshop for the next iteration of design. Looking back, was the requirements workshop with neuroscientists was an evaluation workshop of their existing tools?}

% \ek{Here's another future use of workshops. We've focused on their use in the design process. But, what about their use in research methodologies? The \emph{write} and \emph{reflect} stages of the DSM focus on problem formulation and evaluation of decisions made throughout the design process. Writing the paper is \emph{``time to revisit abstractions, to identify contributions, and to come up with a coherent and understandable line of argumentation.''} Workshops could be valuable for this, particularly in long term reflective collaborations where coauthors need to reach consensus on the contribution and how they are presented.}

%Further thinking on our reflective analysis reveals questions about the roles of reflection and documentation in visualization. Although we maintained documentation of our collaboration in an adhoc fashion, this was critically important for our analysis and writing. In particular, shared documents evolved with our understanding of workshops and were continuously revised with our opinions, experience descriptions, and literature references as we applied the four metaphorical lenses~\cite{Brookfield1998} in this research. Moreover, we referred to documentation from workshops spanning more than 10 years --- this analysis would have been impossible without well-preserved records. We have assembled the results of our analysis into an audit trail and hope this inspires future thinking on how to document and preserve the decisions through visualization collaborations. This could be particularly useful to document the experiences of future visualization creativity workshops. While we present this approach as a means of establishing knowledge about visualization in practice through evidence generated in multiple diverse studies, we would also like to better understand and apply the methods that are most appropriate for future reflective analysis.


% \ek{I will add a discussion about the confidence in our considerations and recommendations. Essentially, we say consideration to mean 'observation' or 'something to think about.' This is important when talking about workshop design because it's really hard to make definitive statements about what methods to use or even to establish a vocabulary for talking about methods.}


% \ek{We say recommendation to mean 'likely beneficial course of action.' But, some of our recommendations are saying 'you need to think about this!' as in 'it is likely beneficial for you to think about how scripted or improvised the workshop will be [\ref{rec:improvise}].' This implies that we are uncertain about what is the likely beneficial action. In those cases, I've tried to describe what we have done in the past, but we can't really say anything about which is better because it is context dependent. So the language of the recommendation conveys things that we are uncertain about.}

% This section discusses our choice of critically reflective practice and how that methodology may have influenced the resulting framework. We also reflect on how the ideas presented in this paper will influence how we use and report on future workshops.

% Our decision to apply critically reflective practice arose from necessity. Quantitative experiments were not appropriate for our applied work \ek{needs citation}. True, workshops produce qualitative data---artifacts and documentation. But qualitative analysis methods, such as thematic analysis and grounded theory, are limited by the scope of these data which do not necessarily capture experiential knowledge. We used critically reflective practice to embrace the subjective nature of using workshops and to articulate tacit knowledge from workshop experience. Insights presented in this paper are inherently subjective --- for example, the cascading influence of workshop methods on final visualization designs (in Sec.~\ref{sec:design}) was apparent only when we considered the workshop data in the larger context of our experiential knowledge as researchers. Also, the motivation of our actions were not apparent in our data as in the amount of improvisation in workshop design (Sec.~\ref{sec:recommendations}) or decisions to show certain graphics during the Visualization Analogies method (Sec.~\ref{sec:design}). These insights could only have resulted from considering the qualitative workshop data in combination with our experience. 

% Although critically reflective practice is inherently subjective, we believe the results of our research are valid because of our diverse backgrounds and experience. We have backgrounds in GIS, software engineering, real-time rendering, and design methodologies, as well as varying levels of experience, ranging from graduate students, to senior researchers. Our framework conveys our agreements (and disagreements) on our subjective interpretation of diverse workshop experience, including collaborations with 8 distinct domains spanning three continents. It was generated through diverse analysis methods, including theming workshop artifacts, collaborative writing, discussion, and observations-to-insights --- which are all documented in our supplemental material. Although inductive research, such as the descriptive framework in this paper, cannot achieve predictive validity, we believe that our framework is valuable in that it describes such diverse experience and provides guidance to visualization researchers who want to use workshops in their own projects. 

%Yet, we consider this framework as an early step towards understanding how and why to use visualization creativity workshops. We have intentionally focused on the visualization workshop design and process at a high level of abstraction in order to identify areas of workshop literature and our experience that are relevant to the visualization community while embracing the flexible nature of workshops and encouraging their use in ways we cannot predict. We intend for our framework to complement existing workshop theory, particularly when considering the details of workshops such as handling logistics, facilitating group discussions, and managing potential conflicts. Three excellent resources for workshops that complement our framework are books by Brooks-Harris and Stock-Ward~\cite{Brooks-Harris1999}, Hamilton~\cite{Hamilton2016}, and Gray~\cite{Macanufo2010}.

%A potential limitation of our framework is that our experience has rlied on collaboration between visualization researchers and analysts who work together to explore visualization solutions to domain challenges. The analysts are motivated to better understand their data in order to advance their personal work or their field as a whole. While this is a common type of user in design studies, it is also interesting to consider visualization design for casual users who may appreciate visualizations for their aesthetic or playful qualities~\cite{McCurdy2016}. We suspect that aspects of our framework may be relevant to casual users, but this is an interesting area for future work. \ek{Maybe we can say something about participatory design?}




% \ek{We need to talk about what goes in this section.}

% This section discusses the tradeoffs of research based in reflection, describes the intended use of ideas in this paper, compares creativity requirements workshops to other methods, and outlines areas for future work.

% \subsection{Research Process}

% Critically reflective practice is appropriate for analyzing our experiences when compared to other research approaches. It captures experiential knowledge and subjective interpretation of experience that is omitted in grounded theory, thematic analysis and similar qualitative methods. Through rigorous reflective methods, we have reached a consensus on the interpretation of our experiences and agreed on prescriptive recommendations for future workshops.
% \jd{This sounds bold and persuasive -- but I think the evidence that we have in fact been rigorously reflective is a little sparse. Can we make this more apparent?}
% We recognize that prescriptive recommendations do not exhibit predictive validity. This is a common challenge in applied and ecologically valid research, especially where creativity is involved. Creativity relies on intrinsic motivation~\cite{Nickerson1999}, which can be hard to replicate in controlled environments for laboratory experiments. 

% \subsection{Intended Use of this Framework}

% We intend for this framework to provide descriptive language about the intent of workshops, workshop methods, and workshop analysis. All recommendations are meant to describe a likely beneficial course of action based on our experience. They are not predictive. Nor do they exhaustively describe all the characteristics of creativity workshops. In fact, one strength of creativity workshops is that they are a flexible method that can fulfill many roles in the design process. Our framework should be used in a way that supports the divergent use of creativity workshops---celebrating their flexibility and exploring their possibilities.
% \jd{So, again, I think we need scope for documenting effects and providing feedback to us and others on experiences to be emphasized and facilitated here somewhere. E.g. R.22 - Gain feedback on workshop processes  form participants and use this to improve knowledge of workshop operation and inform future workshop planning.}

% \subsection{Future work}

% We focused our collective reflection and analysis on creativity requirements workshops, used for the \emph{understand} and \emph{ideate} design activities. We hope to continue this analysis to describe our experience using workshops for the \emph{ideate} and \emph{make} design activities too.